\newcommand{\Eq}{\mathsf{Eq}}
\newcommand{\fib}[2]{\ensuremath{\mathsf{fib}({#1},{#2})}} % let's avoid f^{-1}(y), since that also denotes inverse in categories
\newcommand{\iscontr}{\mathsf{isContr}}
\newcommand{\istype}[1]{\mathsf{is}\mbox{-}{#1}\mbox{-}\mathsf{type}}
\newcommand{\idtoeq}{\ensuremath{\mathsf{idtoeq}}}

\section{Background on Univalent Foundations}
\label{sec:background}

In this section we gather some basic definitions and facts from the Univalent Foundations, in order to fix the terminology used
later on. 
An introduction to the Univalent Foundations can be found in \cite{pelayo-warren:univalent-foundations-paper}.

\begin{defn}[see {\cite[Sec.~7.1]{pelayo-warren:univalent-foundations-paper}}]
 A type $X$ is \textbf{contractible} if there exists $x_0 : X$ such that for any $x : X$ we have $\id[X]{x_0}{x}$.
\end{defn}

\begin{defn}[see {\cite[Sec.~8.3]{pelayo-warren:univalent-foundations-paper}}]\label{def:hlevel}
The \textbf{$n$-types} are defined by recursion as follows:
\begin{itemize}
 \item A type $X$ is a $(-2)$-type if it is contractible.
 \item A type $X$ is a $(n+1)$-type if for all $x, y : X$ the type $\id[X]x y$ is an $n$-type.
\end{itemize}
\end{defn}

The $(-1)$-types will be called \textbf{propositions} and the $0$-types \textbf{sets}.

We next turn towards the notion of equivalence of types, for which we need an intermediate notion.

\begin{defn}[see {\cite[Sec.~7.2]{pelayo-warren:univalent-foundations-paper}}]
 Given a map $f \colon X \to Y$ and an element $y : Y$ the \textbf{fiber} of $f$ over $y$ is the type
 \[\fib{f}{y} := \sum_{x : X} \id[Y]{fx}{y}\]
\end{defn}

\begin{defn}[see {\cite[Sec.~7.3]{pelayo-warren:univalent-foundations-paper}}]
 A map $f \colon X \to Y$ is an \textbf{equivalence} if for every $y : Y$, the homotopy fiber $\fib{f}{y}$ is contractible.
\end{defn}
Note that being an equivalence is a mere proposition.

The type of all equivalences between $X$ and $Y$ will be denoted $\Eq(X, Y)$. The following lemma gives a useful way of proving that a map is an equivalence:

\begin{lem}[see {\cite[Sec.~7.4]{pelayo-warren:univalent-foundations-paper}}]
 A map $f \colon X \to Y$ is an equivalence if and only if it is a \textbf{homotopy equivalence} in the sense that there exists $g \colon Y \to X$ together with:
 \[\eta : \prod_{x : X} (\id{x}{gf(x)}) \qquad \text{and} \qquad \epsilon : \prod_{y : Y} (\id{y}{fg(y)})\]
\end{lem}

The type of equivalences and homotopy equivalences between any two types are not themselves equivalent. 
While ``being an equivalence'' is a mere proposition, ``being a homotopy equivalence'' is not. 
In other words, the map taking a homotopy equivalence between $X$ and $Y$ to its underlying map is not an inclusion.

For a type universe $\type$, given $X, Y : \type$, by induction on identity, we may construct a map
\begin{equation} \label{eq:ua_map}
 \idtoeq_{X,Y} : (\id X Y) \to \Eq(X, Y).
\end{equation}

\begin{defn}[Univalence Axiom, see {\cite[Sec.~8.1]{pelayo-warren:univalent-foundations-paper}}]
 A universe $\type$ is \textbf{univalent}, if for any $X, Y : \type$, the map $\idtoeq_{X,Y}$ of \autoref{eq:ua_map} is an equivalence.
\end{defn}
In the following we work in a hierarchy of \emph{univalent} universes as described above.

