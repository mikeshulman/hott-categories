\documentclass{amsart}
\usepackage[utf8]{inputenc}
\usepackage{amssymb,amsmath,amsthm,stmaryrd,mathrsfs}
\usepackage{enumitem,mathtools,xspace,xcolor}
\definecolor{darkgreen}{rgb}{0,0.45,0}
\definecolor{dkblue}{rgb}{0,0.1,0.5}
\definecolor{lightblue}{rgb}{0,0.5,0.5}
\definecolor{dkgreen}{rgb}{0,0.4,0}
\definecolor{dk2green}{rgb}{0.4,0,0}
\definecolor{dkviolet}{rgb}{0.6,0,0.8}
\usepackage{listings}
\def\lstlanguagefiles{defManSSR.tex}
\lstset{language=SSR}
\usepackage[pagebackref,colorlinks,citecolor=darkgreen,linkcolor=darkgreen]{hyperref}
\usepackage[all,2cell]{xy}
\UseAllTwocells
\usepackage{natbib}
\usepackage{braket}

%%%% MACROS FOR NOTATION %%%%
% Use these for any notation where there are multiple options.

%%% Definitional equality (used infix) %%%
\newcommand{\jdeq}{\equiv}      % An equality judgment
\let\judgeq\jdeq
\newcommand{\defeq}{\coloneqq}  % An equality currently being defined

%%% Dependent products %%%
%% Call the macro like \prd{x,y:A}{p:x=y} with any number of
%% arguments.  Make sure that whatever comes *after* the call doesn't
%% begin with an open-brace, or it will be parsed as another argument.
\makeatletter
%% This version does {\textstyle\prod}(x,y:A)(p:x=y),
% \def\prd{{\textstyle\prod}\@prd}
% \def\@prd#1{(#1)\@ifnextchar\bgroup{\@prd}{,}}
%% This version does {\textstyle\prod}(x,y:A)\;{\textstyle\prod}(p:x=y),
\def\prd#1{{\textstyle\prod}(#1)\@ifnextchar\bgroup{\;\prd}{,}}
%% This version does ({\textstyle\prod}x,y:A)\;({\textstyle\prod}p:x=y),
%\def\prd#1{\big({\textstyle\prod} #1\big)\@ifnextchar\bgroup{\;\prd}{,}}
%% This version does \prod_{(x,y:A)} \prod_{(p:x=y)}
% \def\prd#1{\prod_{(#1)}\@ifnextchar\bgroup{\prd}{}}
%% This version does {\textstyle\prod}_{(x,y:A)} {\textstyle\prod}_{(p:x=y)}
% \def\prd#1{{\textstyle\prod}_{(#1)}\@ifnextchar\bgroup{\prd}{}}
%% This one is Agda style
% \def\prd#1{(#1)\to\@ifnextchar\bgroup{\prd}{}}

%%% Dependent products written with \forall, in the same style
\def\fall#1{{\forall}(#1)\@ifnextchar\bgroup{\;\fall}{,}}

%%% Dependent sums %%%
\def\sm#1{{\textstyle\sum}(#1)\@ifnextchar\bgroup{\;\prd}{,}}
\makeatother
\let\setof\Set    % from package 'braket', write \setof{ x:A | P(x) }.

%%% Identity types %%%
\newcommand{\id}[3][]{\ensuremath{#2 =_{#1} #3}\xspace}
\newcommand{\idtype}[3][]{\ensuremath{\mathsf{Id}_{#1}(#2,#3)}\xspace}

%%% Reflexivity terms %%%
\newcommand{\refl}[1]{\ensuremath{\mathsf{refl}_{#1}}\xspace}

%%% Path concatenation (used infix, in diagrammatic order) %%%
\newcommand{\ct}{\mathrel{\raisebox{.5ex}{$\centerdot$}}}

%%% Path reversal %%%
\newcommand{\opp}[1]{\mathord{{#1}^{-1}}}
\let\rev\opp

%%% Transport (covariant) %%%
\newcommand{\trans}[2]{\ensuremath{{#1}_{*}\!\left({#2}\right)}\xspace}
\newcommand{\transf}[1]{\ensuremath{{#1}_{*}}\xspace} % Without argument
\newcommand{\transport}[2]{\ensuremath{\mathsf{transport}_{*} \: {#2}\xspace}}
\newcommand{\transfib}[3]{\ensuremath{\mathsf{transport}^{#1}(#2,#3)\xspace}}

%%% Map on paths %%%
\newcommand{\mapfunc}[1]{\ensuremath{\mathsf{ap}_{#1}}\xspace} % Without argument
\newcommand{\map}[2]{\ensuremath{{#1}\left({#2}\right)}\xspace}
\newcommand{\mapdep}[2]{\ensuremath{{#1}\llparenthesis{#2}\rrparenthesis}\xspace}
\let\ap\map
\let\apd\mapdep

%%% Identity functions %%%
\newcommand{\idfunc}[1][]{\ensuremath{\mathsf{id}_{#1}}\xspace}

%%% Equivalence types %%%
\newcommand{\eqv}[2]{\ensuremath{#1 \simeq #2}\xspace}

%%% Universe types %%%
\newcommand{\type}{\ensuremath{\mathsf{Type}}\xspace}
\renewcommand{\set}{\ensuremath{\mathsf{Set}}\xspace}
\newcommand{\prop}{\ensuremath{\mathsf{Prop}}\xspace}

%%% Projections out of dependent sums %%%
\newcommand{\proj}[1]{\ensuremath{\mathsf{pr}_{#1}}\xspace}

%%% Bracket/squash/truncation types %%%
% \newcommand{\brck}[1]{\textsf{mere}(#1)}
% \newcommand{\Brck}[1]{\textsf{mere}\Big(#1\Big)}
\newcommand{\trunc}[2]{\tau_{#1}(#2)}
\newcommand{\Trunc}[2]{\tau_{#1}\Big(#2\Big)}
\newcommand{\truncf}[1]{\tau_{#1}}
\def\pizero{\trunc0}
\newcommand{\brck}[1]{\trunc{-1}{#1}}
\newcommand{\Brck}[1]{\Trunc{-1}{#1}}

%%% The unit type
\newcommand{\unit}{\ensuremath{\mathbf{1}}\xspace}
\newcommand{\ttt}{\ensuremath{\star}\xspace}

%%% Injections into binary sums
\newcommand{\inl}{\ensuremath{\mathsf{inl}}\xspace}
\newcommand{\inr}{\ensuremath{\mathsf{inr}}\xspace}

%%% Blanks (i.e. anonymous lambdas)
\newcommand{\blank}{\mathord{\underline{\hspace{1ex}}}}
%\newcommand{\blank}{(-)}
%\newcommand{\blank}{(?)}

%%% Some decorations
\newcommand{\bbU}{\ensuremath{\mathbb{U}}\xspace}
\newcommand{\bbB}{\ensuremath{\mathbb{B}}\xspace}
\newcommand{\bbP}{\ensuremath{\mathbb{P}}\xspace}

%%% Some categories
\newcommand{\uset}{\ensuremath{\underline{\set}}\xspace}
\newcommand{\ucat}{\ensuremath{\underline{\mathsf{Cat}}}\xspace}
\newcommand{\urel}{\ensuremath{\underline{\mathsf{Rel}}}\xspace}
\newcommand{\uhilb}{\ensuremath{\underline{\mathsf{Hilb}}}\xspace}

%%% Spheres
\newcommand{\Sn}{\mathbb{S}}

%%% Categorical models
\newcommand{\m}[1]{\llbracket#1\rrbracket}


%%%% THEOREM ENVIRONMENTS %%%%

% Hyperref includes the command \autoref{...} which is like \ref{...}
% except that it automatically inserts the type of the thing you're
% referring to, e.g. it produces "Theorem 3.8" instead of just "3.8"
% (and makes the whole thing a hyperlink).  This saves a slight amount
% of typing, but more importantly it means that if you decide later on
% that 3.8 should be a Lemma or a Definition instead of a Theorem, you
% don't have to change the name in all the places you referred to it.

% The following hack improves on this by using the same counter for
% all theorem-type environments, so that after Theorem 1.1 comes
% Corollary 1.2 rather than Corollary 1.1.  This makes it much easier
% for the reader to find a particular theorem when flipping through
% the document.
\makeatletter
\def\defthm#1#2{%
  %% All types of theorems are numbered inside sections
  \newtheorem{#1}{#2}[section]%
  %% This command tells hyperref's \autoref what to call things
  \expandafter\def\csname #1autorefname\endcsname{#2}%
  %% This makes all the theorem counters secretly the same counter
  \expandafter\let\csname c@#1\endcsname\c@thm}

% Now define a bunch of theorem-type environments.
\newtheorem{thm}{Theorem}[section]
\newcommand{\thmautorefname}{Theorem}
%\defthm{prop}{Proposition}   % Probably we shouldn't use "Proposition" in this way
\defthm{cor}{Corollary}
\defthm{lem}{Lemma}
% Since definitions and theorems in type theory are synonymous, should
% we actually use the same theoremstyle for them?
\theoremstyle{definition}
\defthm{defn}{Definition}
\theoremstyle{remark}
\defthm{rmk}{Remark}
\defthm{eg}{Example}
\defthm{egs}{Examples}
\defthm{notes}{Notes}


%%%% EQUATION NUMBERING %%%%

% The following hack uses the single theorem counter to number
% equations as well, so that we don't have both Theorem 1.1 and
% equation (1.1).
\let\c@equation\c@thm
\numberwithin{equation}{section}


%%%% ENUMERATE NUMBERING %%%%

% Number the first level of enumerates as (i), (ii), ...
\renewcommand{\theenumi}{(\roman{enumi})}
\renewcommand{\labelenumi}{\theenumi}


%%%% MARGINS %%%%

% This is a matter of personal preference, but I think the left
% margins on enumerates and itemizes are too wide.
\setitemize[1]{leftmargin=2em}
\setenumerate[1]{leftmargin=*}

%%%% CITATIONS %%%%

\newcommand{\ra}{\rightarrow}

%\newcommand{\refl}{{\sf r}}
\newcommand{\gives}{{\;\rhd\;}}
\newcommand{\oa}{\overline{a}}

\newcommand{\bbt}{t\! t}
\newcommand{\bbf}{f\!\! f}

%\newcommand{\proof}{{\bf Proof: }}
\newcommand{\MRqed}{\hfill $\Box$\\[1ex]}
%\newcommand{\qed}{\MRqed}


\newcommand{\inv}[1]{{#1}^{-1}}
\newcommand{\idtoiso}{\ensuremath{\mathsf{idtoiso}}\xspace}
\newcommand{\isotoid}{\ensuremath{\mathsf{isotoid}}\xspace}
\newcommand{\op}{^{\textrm{op}}}
\newcommand{\y}{\ensuremath{\mathbf{y}}\xspace}
\newcommand{\dgr}[1]{{#1}^{\dagger}}
\newcommand{\unitaryiso}{\mathrel{\cong^\dagger}}

\title{Univalent categories and the Rezk completion}
\author{Benedikt Ahrens}
\author{Krzysztof Kapulkin}
\author{Michael Shulman}
\date{\today}
\begin{document}
\maketitle

\section{Introduction}
\label{sec:introduction}

Of the branches of mathematics, category theory is one which fits the least comfortably into existing ``foundations of mathematics''.
This is true both at an informal level, and when trying to be completely formal using a computer proof assistant.
One problem is that category theory would like to use constructions such as ``the category of all sets'', which naively conceived, run afoul of Russellian paradoxes and have to be reinterpreted using universe levels.
Another is that most of category theory is invariant under weaker notions of ``sameness'' than equality, such as isomorphism in a category or equivalence of categories, in a way which traditional foundations (such as set theory) fail to capture.
\marginpar{Are there other problems specific to formalization?}

Our aim in this paper is to show that these problems can be ameliorated using the new \emph{univalent foundations} of mathematics, a.k.a.\ \emph{homotopy type theory}.
Univalent foundations is a new approach to foundations of mathematics,
proposed by V. Voevodsky \cite{vv_uf}, building on the existing development
of dependent type theory [reference], a logical system feasible for
(large-scale) formalization of mathematics [references]. The distinctive
feature of the Univalent Foundations is its treatment of equality inspired
by homotopy--theoretic semantics [references]. Voevodsky extended the
dependent type theory by an additional axiom, called the \emph{Univalence
Axiom}, suggested by the interpretation of the theory in the category of
simplicial sets [reference].

The univalence axiom identifies \emph{equivalence} of types with \emph{identity} of types.
In particular, this implies that anything we can say about sets is automatically invariant under isomorphism, because isomorphism is identified with identity.
In other words, under the univalence axiom, the category of sets \emph{automatically} behaves ``categorically'', in that its objects cannot be distinguished more finely than up to (unique) isomorphism.
Our goal in this paper is to extend this behavior to other categories, which requires a more careful analysis of the definition of ``category''.

If we ignore size issues, then in set-based mathematics, a category consists of a \emph{set} of objects and, for each pair $x,y$ of objects, a \emph{set} $\hom(x,y)$ of morphisms.
Under univalent foundations, a ``naive'' definition of category would simply mimic this with a \emph{type} of objects and \emph{types} of morphisms.
However, if we allowed these types to contain arbitrary higher homotopy, then we ought to impose higher coherence conditions on the associativity and unitality axioms, leading to some notion of $(\infty,1)$-category.
Eventually this should be done, but at present our goal is more modest.
We restrict ourselves to 1-categories, and therefore we restrict the hom-types $\hom(x,y)$ to be \emph{sets} in the sense of UF, i.e.\ types satisfying UIP.

More interesting is whether the type of objects should have any higher homotopy.
If we require it also to be a set, then we end up with a definition that behaves more like the traditional set-theoretic one.
Following Toby Bartels, we call this notion a \emph{strict category}.
However, a (usually) better option is to require it to satisfy a generalized version of the univalence axiom, identifying the \emph{identity type} $(x=_{\mathsf{Obj}} y)$ between two objects with the type $\mathsf{iso}(x,y)$ of \emph{isomorphisms} from $x$ to $y$.
(In particular, this implies that each type $(x=_{\mathsf{Obj}} y)$ is a set, and that therefore the type of objects is a \emph{1-type}, containing no higher homotopy above dimension 1.)
We consider this to be the ``correct'' definition of \emph{category} in univalent foundations; for emphasis we may call it a \emph{saturated} or \emph{univalent} category.

The categories we often encounter in practice are not saturated (such as the category on $n$-types).
As we will see in the next sections, they behave much worse than the saturated ones.
This is why we call them \emph{precategories} to emphasize that the theory should rather be developed for saturated categories.
In fact, we can see that with many respects the set-theoretic category theory is reflected by saturated categories, not precategories.
And so we will drop the adjective ``saturated'' and refer to them as simply ``categories''

There is an obvious inclusion, i.e. a map with fibers being propositions, of (univalent) categories into precategories.
One of the main goals of this paper is to show that this inclusion has a left adjoint (in the appropriate bicategorical sense).
That is, given any precategory $A$, we construct a (univalent) category $\hat{A}$ together with a fully faithful and essentially surjective functor
(the unit of the adjunction) $A \to \hat{A}$.

Our construction is a type-theoretic analog of the familiar homotopy-theoretic construction given by Rezk \cite[Sec.~14]{rezk01css} for Segal spaces.

We consider types in Univalent Foundations to be stratified in a
linearly ordered hierarchy of \emph{universes}.
Throughout the paper we avoid mentioning universes explicitly,
appealing instead to \emph{typical ambiguity}:
type variables live in \emph{some} universe, and universes 
are quantified over implicitly.
All the constructions of this paper can be carried out in \emph{one}
universe, with one exception: the Yoneda embedding rises the
universe level by one, since the category of h-sets of universe, say, $n$ lives in universe $n+1$.

In \S\S\ref{sec:cats}--\ref{sec:yoneda} we develop the basic theory of such categories informally: functors, natural transformations, adjunctions, equivalences, and the Yoneda lemma.
Then in \S\ref{sec:rezk} we construct the \emph{Rezk completion}, which makes any precategory into a category in a universal way.

Large parts of this development have been formally verified in the proof assistant \textsf{Coq}, building on Voevodsky's \emph{Foundations} library \cite{vv_foundations}.
In particular, the formalization includes the Rezk completion together with its universal property.
Section \ref{sec:formalization} discusses content of the formalization, organization of source files, and the differences between informal presentation and its formal analog.

Finally, in \S\ref{sec:semantics} we discuss semantics\dots [something about equipments, CSS, stacks]\dots

\dots [conclusion] \dots

\paragraph*{Acknowledgements} First and foremost we would like to thank Vladimir Voevodsky for \ldots [possibly infinite list of things] \ldots [but grant support was also really helpful] \ldots

\section{Categories and precategories}
\label{sec:cats}

In classical mathematics, there are many equivalent definitions of a category.
In our case, since we have dependent types, it is natural to choose the arrows to be dependent over the objects.
This matches the way hom-types are always used in category theory: we never even consider comparing two arrows unless we know their sources and targets agree.
Furthermore, it seems clear that for a theory of 1-categories, the hom-types should all be sets.
This leads us to the following definition.

\begin{defn}\label{ct:precategory}
  A \textbf{precategory} $A$ consists of the following.
  \begin{enumerate}
  \item A type $A_0$ of \emph{objects}.  We write $a:A$ for $a:A_0$.
  \item For each $a,b:A$, a set $\hom_A(a,b)$ of \emph{arrows} or \emph{morphisms}.
  \item For each $a:A$, a morphism $1_a:\hom_A(a,a)$.
  \item For each $a,b,c:A$, a function
    \[  \hom_A(b,c) \to \hom_A(a,b) \to \hom_A(a,c) \]
    denoted infix by $g\mapsto f\mapsto g\circ f$, or sometimes simply by $gf$.
  \item For each $a,b:A$ and $f:\hom_A(a,b)$, we have $\id f {1_b\circ f}$ and $\id f {f\circ 1_a}$.
  \item For each $a,b,c,d:A$ and $f:\hom_A(a,b)$, $g:\hom_A(b,c)$, $h:\hom_A(c,d)$, we have $\id {h\circ (g\circ f)}{(h\circ g)\circ f}$.
  \end{enumerate}
\end{defn}

The problem with the notion of precategory is that for objects $a,b:A$, we have two possibly-different notions of ``sameness''.
On the one hand, we have $\id[A_0]{a}{b}$.
But on the other hand, there is the standard categorical notion of \emph{isomorphism}.

\begin{defn}\label{ct:isomorphism}
  A morphism $f:\hom_A(a,b)$ is an \textbf{isomorphism} if there is a morphism $g:\hom_A(b,a)$ such that $\id{g\circ f}{1_a}$ and $\id{f\circ g}{1_b}$.
  We write $a\cong b$ for the type of such isomorphisms.
\end{defn}

\begin{lem}\label{ct:isoprop}
  For any $f:\hom_A(a,b)$, the type ``$f$ is an isomorphism'' is a mere proposition.
  Therefore, for any $a,b:A$ the type $a\cong b$ is a set.
\end{lem}
\begin{proof}
  Suppose given $g:\hom_A(b,a)$ and $\eta:(\id{1_a}{g\circ f})$ and $\epsilon:(\id{f\circ g}{1_b})$, and similarly $g'$, $\eta'$, and $\epsilon'$.
We must show $\id{(g,\eta,\epsilon)}{(g',\eta',\epsilon')}$.
  But since all hom-sets are sets, their identity types are mere propositions, so it suffices to show $\id g {g'}$.
  For this we have
  \[g' = 1_a\circ g' = (g\circ f)\circ g' = g\circ (f\circ g') = g\circ 1_b = g\]
  using $\eta$ and $\epsilon'$.
\end{proof}

If $f:a\cong b$, then we write $\inv f$ for its inverse, which by \autoref{ct:isoprop} is uniquely determined.

The only relationship between these two notions of sameness that we have in a precategory is the following.

\begin{lem}[\textsf{idtoiso}]\label{ct:idtoiso}
  If $A$ is a precategory and $a,b:A$, then
  \[(\id a b)\to (a \cong b).\]
\end{lem}
\begin{proof}
  By induction on identity, we may assume $a$ and $b$ are the same.
  But then we have $1_a:\hom_A(a,a)$, which is clearly an isomorphism.
\end{proof}

Evidently, this situation is analogous to the issue that motivated us to introduce the univalence axiom.
In fact, we have the following:

\begin{eg}\label{ct:precatset}
  There is a precategory \uset, whose type of objects is \set, and with $\hom_{\uset}(A,B) \defeq (A\to B)$.
  The identity morphisms are identity functions and the composition is function composition.
  For this precategory, \autoref{ct:idtoiso} is equal to (the restriction to sets of) the identity-to-equivalence map from the chapter on univalence.
\end{eg}

Thus, it is natural to make the following definition.

\begin{defn}\label{ct:category}
  A \textbf{category} is a precategory such that for all $a,b:A$, the function $\idtoiso_{a,b}$ from \autoref{ct:idtoiso} is an equivalence.
\end{defn}

In particular, in a category, if $a\cong b$, then $a=b$.
The usefulness of a definition of categories of this sort, where the type of objects contains all the categorical isomorphisms, was first strongly emphasized (in the context of set-based definitions of higher categories) by Charles Rezk~\cite{rezk01css}.

\begin{eg}\label{ct:eg:set}
  The univalence axiom implies immediately that \uset is a category.
  One can also show, using univalence, that any precategory of set-level structures such as groups, rings, topological spaces, etc.\ is a category.
\end{eg}

We also note the following.

\begin{lem}\label{ct:obj-1type}
  In a category, the type of objects is a 1-type.
\end{lem}
\begin{proof}
  It suffices to show that for any $a,b:A$, the type $\id a b$ is a set.
  But $\id a b$ is equivalent to $a \cong b$, which is a set.
\end{proof}

We write $\isotoid$ for the inverse $(a\cong b) \to (\id a b)$ of the map $\idtoiso$ from \autoref{ct:idtoiso}.
The following relationship between the two is important.

\begin{lem}\label{ct:idtoiso-trans}
  For $p:\id a a'$ and $q:\id b b'$ and $f:\hom_A(a,b)$, we have
  \begin{equation}\label{ct:idtoisocompute}
    \id{\trans{(p,q)}{f}}
    {\idtoiso(q)\circ f \circ \inv{\idtoiso(p)}}
  \end{equation}
\end{lem}
\begin{proof}
  By induction, we may assume $p$ and $q$ are $\refl a$ and $\refl b$ respectively.
Then the left-hand side of~\eqref{ct:idtoisocompute} is simply $f$.
  But by definition, $\idtoiso(\refl a)$ is $1_a$, and $\idtoiso(\refl b)$ is $1_b$, so the right-hand side of~\eqref{ct:idtoisocompute} is $1_b\circ f\circ 1_a$, which is equal to $f$.
\end{proof}

Similarly, we can show
\begin{gather}
  \id{\idtoiso(\rev p)}{\inv {(\idtoiso(p))}}\\
  \id{\idtoiso(p\ct q)}{\idtoiso(q)\circ \idtoiso(p)}\\
  \id{\isotoid(f\circ e)}{\isotoid(e)\ct \isotoid(f)}
\end{gather}
and so on.

\begin{eg}\label{ct:orders}
  A precategory in which each set $\hom_A(a,b)$ is a mere proposition is equivalently a type $A_0$ equipped with a mere relation ``$\le$'' that is reflexive ($a\le a$) and transitive (if $a\le b$ and $b\le c$, then $a\le c$).
  We call this a \textbf{preorder}.

  In a preorder, a proof $f\colon a\le b$ is an isomorphism just when there exists some proof $g\colon b\le a$.
  Thus, $a\cong b$ is the mere proposition that $a\le b$ and $b\le a$.
  Therefore, a preorder $A$ is a category just when (1) each type $a=b$ is a mere proposition, and (2) for any $a,b:A_0$ there exists a function $(a\cong b) \to (a=b)$.
  In other words, $A_0$ must be a set, and $\le$ must be antisymmetric (if $a\le b$ and $b\le a$, then $a=b$).
  We call this a \textbf{(partial) order} or a \textbf{poset}.
\end{eg}

\begin{eg}\label{ct:gaunt}
  If $A$ is a category, then $A_0$ is a set if and only if for any $a,b:A_0$, the type $a\cong b$ is a mere proposition.
  This is equivalent to saying that every isomorphism in $A$ is an identity; thus it is rather stronger than the classical notion of ``skeletal'' category.
  Categories of this sort are sometimes called \textbf{gaunt} (this term was introduced by Barwick and Schommer-Pries~\cite{bsp12infncats}).
  There is not really any notion of ``skeletality'' for our categories, unless one considers \autoref{ct:category} itself to be such.
\end{eg}

\begin{eg}\label{ct:discrete}
  For any 1-type $X$, there is a category with $X$ as its type of objects and with $\hom(x,y) \defeq (x=y)$.
  If $X$ is a set, we call this the \textbf{discrete} category on $X$.
  In general, we call it a \textbf{groupoid}.
\end{eg}

\begin{eg}\label{ct:fundgpd}
  For \emph{any} type $X$, there is a precategory with $X$ as its type of objects and with $\hom(x,y) \defeq \pizero{x=y}$.
  The composition operation
  \[ \pizero{y=z} \to \pizero{x=y} \to \pizero{x=z} \]
  is defined by induction on truncation from concatenation $(y=z)\to(x=y)\to(x=z)$.
  We call this the \emph{fundamental pregroupoid} of $X$.
\end{eg}

\begin{eg}\label{ct:hoprecat}
  There is a precategory whose type of objects is \type and with $\hom(X,Y) \defeq \pizero{X\to Y}$, and composition defined by induction on truncation from ordinary composition $(Y\to Z) \to (X\to Y) \to (X\to Z)$.
  We call this the \emph{homotopy precategory of types}.
\end{eg}

\begin{eg}\label{ct:rel}
  Let \urel be the following precategory:
  \begin{itemize}
  \item Its objects are sets.
  \item $\hom_{\urel}(X,Y) = X\to Y\to \prop$.
  \item For a set $X$, we have $1_X(x,x') \defeq (x=x')$.
  \item For $R:\hom_{\urel}(X,Y)$ and $S:\hom_{\urel}(Y,Z)$, their composite is defined by
    \[ (S\circ R)(x,z) \defeq \Brck{\sm{y:Y} R(x,y) \times S(y,z)}.\]
  \end{itemize}
  Suppose $R:\hom_{\urel}(X,Y)$ is an isomorphism, with inverse $S$.
  We observe the following.
  \begin{enumerate}
  \item If $R(x,y)$ and $S(y',x)$, then $(R\circ S)(y',y)$, and hence $y'=y$.
    Similarly, if $R(x,y)$ and $S(y,x')$, then $x=x'$.\label{item:rel1}
  \item For any $x$, we have $x=x$, hence $(S\circ R)(x,x)$.
    Thus, there merely exists a $y:Y$ such that $R(x,y)$ and $S(y,x)$.\label{item:rel2}
  \item Suppose $R(x,y)$.
    By~\ref{item:rel2}, there merely exists a $y'$ with $R(x,y')$ and $S(y',x)$.
    But then by~\ref{item:rel1}, merely $y'=y$, and hence $y'=y$ since $Y$ is a set.
    Therefore, by transporting $S(y',x)$ along this equality, we have $S(y,x)$.
    In conclusion, $R(x,y)\to S(y,x)$.
    Similarly, $S(y,x) \to R(x,y)$.\label{item:rel3}
  \item If $R(x,y)$ and $R(x,y')$, then by~\ref{item:rel3}, $S(y',x)$, so that by~\ref{item:rel1}, $y=y'$.
    Thus, for any $x$ there is at most one $y$ such that $R(x,y)$.
    And by~\ref{item:rel2}, there merely exists such a $y$, hence there exists such a $y$.
  \end{enumerate}
  In conclusion, if $R:\hom_{\urel}(X,Y)$ is an isomorphism, then for each $x:X$ there is exactly one $y:Y$ such that $R(x,y)$, and dually.
  Thus, there is a function $f:X\to Y$ sending each $x$ to this $y$, which is an equivalence; hence $X=Y$.
  With a little more work, we conclude that \urel is a category.
\end{eg}

In a textbook on category theory written for readers who had grown up with informal homotopy type theory, we would probably say very little about precategories from now on, restricting ourselves to the case of categories.
However, since this book has a somewhat different purpose and audience, we will develop many concepts for precategories as well as categories, in order to emphasize how much better-behaved categories are, as compared both to precategories and to ordinary categories in classical mathematics.

We will also see in \S\ref{sec:strict-categories}--\ref{sec:dagger-categories} that in slightly more exotic contexts, there are uses for certain kinds of precategories other than categories, each of which ``fixes'' the equality of objects in different ways.
This emphasizes the ``pre''-ness of precategories: they are the raw material out of which multiple important categorical structures can be defined.


\section{Functors and transformations}
\label{sec:transfors}

The following definitions are fairly obvious, and need no modification.

\begin{defn}\label{ct:functor}
  Let $A$ and $B$ be precategories.
  A \textbf{functor} $F:A\to B$ consists of
  \begin{enumerate}
  \item A function $F_0:A_0\to B_0$, generally also denoted $F$.
  \item For each $a,b:A$, a function $F_{a,b}:\hom_A(a,b) \to \hom_B(Fa,Fb)$, generally also denoted $F$.
  \item For each $a:A$, we have $\id{F(1_a)}{1_{Fa}}$.
  \item For each $a,b,c:A$ and $f:\hom_A(a,b)$ and $g:\hom_B(b,c)$, we have
    \[\id{F(g\circ f)}{Fg\circ Ff}.\]
  \end{enumerate}
\end{defn}

Note that by induction on identity, a functor also preserves \idtoiso.

\begin{defn}\label{ct:nattrans}
  For functors $F,G:A\to B$, a \textbf{natural transformation} $\gamma:F\to G$ consists of
  \begin{enumerate}
  \item For each $a:A$, a morphism $\gamma_a:\hom_B(Fa,Ga)$ (the ``components'').
  \item For each $a,b:A$ and $f:\hom_A(a,b)$, we have $\id{Gf\circ \gamma_a}{\gamma_b\circ Ff}$ (the ``naturality axiom'').
  \end{enumerate}
\end{defn}

Since each type $\hom_B(Fa,Ga)$ is a set, its identity type is a mere proposition.
Thus, the naturality axiom is a mere proposition, so identity of natural transformations is determined by identity of their components.
In particualar, for any $F$ and $G$, the type of natural transformations from $F$ to $G$ is again a set.

Similarly, identity of functors is determined by identity of the functions $A_0\to B_0$ and (transported along this) of the corresponding functions on hom-sets.

\begin{defn}\label{ct:functor-precat}
  For precategories $A,B$, there is a precategory $B^A$ defined by
  \begin{itemize}
  \item $(B^A)_0$ is the type of functors from $A$ to $B$.
  \item $\hom_{B^A}(F,G)$ is the type of natural transformations from $F$ to $G$.
  \end{itemize}
\end{defn}
\begin{proof}
  We define $(1_F)_a\defeq 1_{Fa}$.
  Naturality follows by the unit axioms of a precategory.
  For $\gamma:F\to G$ and $\delta:G\to H$, we define $(\delta\circ\gamma)_a\defeq \delta_a\circ \gamma_a$.
  Naturality follows by associativity.
  Similarly, the unit and associativity laws for $B^A$ follow from those for $B$.
\end{proof}

\begin{lem}\label{ct:natiso}
  A natural transformation $\gamma:F\to G$ is an isomorphism in $B^A$ if and only if each $\gamma_a$ is an isomorphism in $B$.
\end{lem}
\begin{proof}
  If $\gamma$ is an isomorphism, then we have $\delta:G\to F$ that is its inverse.
  By definition of composition in $B^A$, $(\delta\gamma)_a\jdeq \delta_a\gamma_a$ and similarly.
  Thus, $\id{\delta\gamma}{1_F}$ and $\id{\gamma\delta}{1_G}$ imply $\id{\delta_a\gamma_a}{1_{Fa}}$ and $\id{\gamma_a\delta_a}{1_{Ga}}$, so $\gamma_a$ is an isomorphism.

  Conversely, suppose each $\gamma_a$ is an isomorphism, with inverse called $\delta_a$, say.
We define a natural transformation $\delta:G\to F$ with components $\delta$; for the naturality axiom we have
  \[ Ff\circ \delta_a = \delta_b\circ \gamma_b\circ Ff \circ \delta_a = \delta_b\circ Gf\circ \gamma_a\circ \delta_a = \delta_b\circ Gf. \]
  Now since composition and identity of natural transformations is determined on their components, we have $\id{\gamma\delta}{1_F}$ and $\id{\delta\gamma}{1_G}$.
\end{proof}

The following result is fundamental.

\begin{thm}\label{ct:functor-cat}
  If $A$ is a precategory and $B$ is a category, then $B^A$ is a category.
\end{thm}
\begin{proof}
  Let $F,G:A\to B$; we must show that $\idtoiso:(\id{F}{G}) \to (F\cong G)$ is an equivalence.

  To give an inverse to it, suppose $\gamma:F\cong G$ is a natural isomorphism.
  Then for any $a:A$, we have an isomorphism $\gamma_a:Fa \cong Ga$, hence an identity $\isotoid(\gamma_a):\id{Fa}{Ga}$.
  By function extensionality, we have an identity $\bar{\gamma}:\id[(A_0\to B_0)]{F_0}{G_0}$.

  Now since the last two axioms of a functor are mere propositions, to show that $\id{F}{G}$ it will suffice to show that for any $a,b:A$, the functions
  \begin{align*}
    F_{a,b}&:\hom_A(a,b) \to \hom_B(Fa,Fb)\mathrlap{\qquad\text{and}}\\
    G_{a,b}&:\hom_A(a,b) \to \hom_B(Ga,Gb)
  \end{align*}
  become equal when transported along $\bar\gamma$.
  By computation for function extensionality, when applied to $a$, $\bar\gamma$ becomes equal to $\isotoid(\gamma_a)$.
  But by \autoref{ct:idtoiso-trans}, transporting $Ff:\hom_B(Fa,Fb)$ along $\isotoid(\gamma_a)$ and $\isotoid(\gamma_b)$ is equal to the composite $\gamma_b\circ Ff\circ \inv{(\gamma_a)}$, which by naturality of $\gamma$ is equal to $Gf$.

  This completes the definition of a function $(F\cong G) \to (\id F G)$.
  Now consider the composite
  \[ (\id F G) \to (F\cong G) \to (\id F G). \]
  Since hom-sets are sets, their identity types are mere propositions, so to show that two identities $p,q:\id F G$ are equal, it suffices to show that $\id[\id{F_0}{G_0}]{p}{q}$.
  But in the definition of $\bar\gamma$, if $\gamma$ were of the form $\idtoiso(p)$, then $\gamma_a$ would be equal to $\idtoiso(p_a)$ (this can easily be proved by induction on $p$).
  Thus, $\isotoid(\gamma_a)$ would be equal to $p_a$, and so by function extensionality we would have $\id{\bar\gamma}{p}$, which is what we need.

  Finally, consider the composite
  \[(F\cong G)\to  (\id F G) \to (F\cong G). \]
  Since identity of natural transformations can be tested componentwise, it suffices to show that for each $a$ we have $\id{\idtoiso(\bar\gamma)_a}{\gamma_a}$.
  But as observed above, we have $\id{\idtoiso(\bar\gamma)_a}{\idtoiso((\bar\gamma)_a)}$, while $\id{(\bar\gamma)_a}{\isotoid(\gamma_a)}$ by computation for function extensionality.
  Since $\isotoid$ and $\idtoiso$ are inverses, we have $\id{\idtoiso(\bar\gamma)_a}{\gamma_a}$ as desired.
\end{proof}

In particular, naturally isomorphic functors between categories (as opposed to precategories) are equal.

\medskip

We now define all the usual ways to compose functors and natural transformations.

\begin{defn}
  For functors $F:A\to B$ and $G:B\to C$, their composite $G\circ F:A\to C$ is given by
  \begin{itemize}
  \item The composite $(G_0\circ F_0) : A_0 \to C_0$
  \item For each $a,b:A$, the composite
    \[(G_{Fa,Fb}\circ F_{a,b}):\hom_A(a,b) \to \hom_C(GFa,GFb).\]
  \end{itemize}
  It is easy to check the axioms.
\end{defn}

\begin{defn}\label{def:whisker}
  For functors $F:A\to B$ and $G,H:B\to C$ and a natural transformation $\gamma:G\to H$, the composite $(\gamma F):GF\to HF$ is given by
  \begin{itemize}
  \item For each $a:A$, the component $\gamma_{Fa}$.
  \end{itemize}
  Naturality is easy to check.
  Similarly, for $\gamma$ as above and $K:C\to D$, the composite $(K\gamma):KG\to KH$ is given by
  \begin{itemize}
  \item For each $b:B$, the component $K(\gamma_b)$.
  \end{itemize}
\end{defn}

\begin{lem}\label{ct:interchange}
  For functors $F,G:A\to B$ and $H,K:B\to C$ and natural transformations $\gamma:F\to G$ and $\delta:H\to K$, we have
  \[\id{(\delta G)(H\gamma)}{(K\gamma)(\delta F)}.\]
\end{lem}
\begin{proof}
  It suffices to check componentwise: at $a:A$ we have
  \begin{align*}
    ((\delta G)(H\gamma))_a
    &\jdeq (\delta G)_{a}(H\gamma)_a\\
    &\jdeq \delta_{Ga}\circ H(\gamma_a)\\
    &= K(\gamma_a) \circ \delta_{Fa} \hspace{2cm}\text{(by naturality of $\delta$)}\\
    &\jdeq (K \gamma)_a\circ (\delta F)_a\\
    &\jdeq ((K \gamma)(\delta F))_a.\qedhere
  \end{align*}
\end{proof}

Classically, one defines the ``horizontal composite'' of $\gamma:F\to G$ and $\delta:H\to K$ to be the common value of ${(\delta G)(H\gamma)}$ and ${(K\gamma)(\delta F)}$.
We will refrain from doing this, because while equal, these two transformations are not \emph{definitionally} equal.
This also has the consequence that we can use the symbol $\circ$ (or juxtaposition) for all kinds of composition unambiguously: there is only one way to compose two natural transformations (as opposed to composing a natural transformation with a functor on either side).

\begin{lem}\label{ct:functor-assoc}
  Composition of functors is associative: $\id{H(GF)}{(HG)F}$.
\end{lem}
\begin{proof}
  Since composition of functions is associative, this follows immediately for the actions on objects and on homs.
  And since hom-sets are sets, the rest of the data is automatic.
\end{proof}

The equality in \autoref{ct:functor-assoc} is likewise not definitional.
(Composition of functions is definitionally associative, but the axioms that go into a functor must also be composed, and this breaks definitional associativity.)  For this reason, we need also to know about \emph{coherence} for associativity.

\begin{lem}\label{ct:pentagon}
  \autoref{ct:functor-assoc} is coherent, i.e.\ the following pentagon of equalities commutes:
  \[ \xymatrix{ & K(H(GF)) \ar[dl] \ar[dr]\\
    (KH)(GF) \ar[d] && K((HG)F) \ar[d]\\
    ((KH)G)F && (K(HG))F \ar[ll] }
  \]
\end{lem}
\begin{proof}
  As in \autoref{ct:functor-assoc}, this is evident for the actions on objects, and the rest is automatic.
\end{proof}

We will henceforth abuse notation by writing $H\circ G\circ F$ or $HGF$ for either $H(GF)$ or $(HG)F$, transporting along \autoref{ct:functor-assoc} whenever necessary.
We have a similar coherence result for units.

\begin{lem}\label{ct:units}
  For a functor $F:A\to B$, we have equalities $\id{(1_B\circ F)}{F}$ and $\id{(F\circ 1_A)}{F}$, such that given also $G:B\to C$, the following triangle of equalities commutes.
  \[ \xymatrix{
    G\circ (1_B \circ F) \ar[rr] \ar[dr] &&
    (G\circ 1_B)\circ F \ar[dl] \\
    & G \circ F.}
  \]
\end{lem}

If we went on to develop higher category theory, we could show that categories, functors, and natural transformations form a 2-category (and not merely a bicategory or a pre-2-category).


\section{Adjunctions}
\label{sec:adjunctions}

\begin{defn}\label{def:adjoint}
  A functor $F:A\to B$ is a \textbf{left adjoint} if there exists
  \begin{itemize}
  \item A functor $G:B\to A$.
  \item A natural transformation $\eta:1_A \to GF$.
  \item A natural transformation $\epsilon:FG\to 1_B$.
  \item $\id{(\epsilon F)(F\eta)}{1_F}$.
  \item $\id{(G\epsilon)(\eta G)}{1_G}$.
  \end{itemize}
\end{defn}

\begin{lem}\label{ct:adjprop}
  If $A$ is a category (but $B$ may be only a precategory), then the type ``$F$ is a left adjoint'' is a mere proposition.
\end{lem}
\begin{proof}
  Suppose given $(G,\eta,\epsilon)$ with the triangle identities and also $(G',\eta',\epsilon')$.
  Define $\gamma:G\to G'$ to be $(G'\epsilon)(\eta' G)$, and $\delta:G'\to G$ to be $(G\epsilon')(\eta G')$.
  Then
  \begin{align*}
    \delta\gamma &=
    (G\epsilon')(\eta G')(G'\epsilon)(\eta'G)\\
    &= (G\epsilon')(G F G'\epsilon)(\eta G' F G)(\eta'G)\\
    &= (G\epsilon)(G\epsilon'FG)(G F \eta' G)(\eta G)\\
    &= (G\epsilon)(\eta G)\\
    &= 1_G
  \end{align*}
  using \autoref{ct:interchange} and the triangle identities.
  Similarly, we show $\id{\gamma\delta}{1_{G'}}$, so $\gamma$ is a natural isomorphism $G\cong G'$.
  By \autoref{ct:functor-cat}, we have an identity $\id G {G'}$.

  Now we need to know that when $\eta$ and $\epsilon$ are transported along this identity, they become equal to $\eta'$ and $\epsilon'$.
  By \autoref{ct:idtoiso-trans}, this transport is given by composing with $\gamma$ or $\delta$ as appropriate.
  For $\eta$, this yields
  \begin{equation*}
    (G'\epsilon F)(\eta'GF)\eta
    = (G'\epsilon F)(G'F\eta)\eta'
    = \eta'
  \end{equation*}
  using \autoref{ct:interchange} and the triangle identity.
  The case of $\epsilon$ is similar.
  Finally, the triangle identities transport correctly automatically, since hom-sets are sets.
\end{proof}

In \S\ref{sec:yoneda} we will give another proof of \autoref{ct:adjprop}.


\section{Equivalences}
\label{sec:equivalences}

\begin{defn}
  A functor $F:A\to B$ is an \textbf{equivalence of (pre)categories} if it is a left adjoint for which $\eta$ and $\epsilon$ are isomorphisms.
  We write $A\simeq B$ for the type of equivalences of categories from $A$ to $B$.
\end{defn}

By \autoref{ct:adjprop} and \autoref{ct:isoprop}, if $A$ is a category, then the type ``$F$ is an equivalence of precategories'' is a mere proposition.

\begin{lem}\label{ct:adjointification}
  If for $F:A\to B$ there exists $G:B\to A$ and isomorphisms $GF\cong 1_A$ and $FG\cong 1_B$, then $F$ is an equivalence of precategories.
\end{lem}
\begin{proof}
  Just like the ``adjointification'' theorem for equivalences of types.
\end{proof}

\begin{defn}
  We say a functor $F:A\to B$ is \textbf{faithful} if for all $a,b:A$, the function
  \[F_{a,b}:\hom_A(a,b) \to \hom_B(Fa,Fb)\]
  is injective, and \textbf{full} if for all $a,b:A$ this function is surjective.
  If it is both (hence each $F_{a,b}$ is an equivalence) we say $F$ is \textbf{fully faithful}.
\end{defn}

\begin{defn}
  We say a functor $F:A\to B$ is \textbf{split essentially surjective} if for all $b:B$ there exists an $a:A$ such that $Fa\cong b$.
\end{defn}

\begin{lem}\label{ct:ffeso}
  For any precategories $A$ and $B$ and functor $F:A\to B$, the following types are equivalent.
  \begin{enumerate}
  \item $F$ is an equivalence of precategories.\label{item:ct:ffeso1}
  \item $F$ is fully faithful and split essentially surjective.\label{item:ct:ffeso2}
  \end{enumerate}
\end{lem}
\begin{proof}
  Suppose $F$ is an equivalence of precategories, with $G,\eta,\epsilon$ specified.
  Then we have the function
  \begin{equation*}
    \begin{array}{rcl}
      \hom_B(Fa,Fb) &\to& \hom_A(a,b)\\
      g &\mapsto& \inv{\eta_b}\circ G(g)\circ \eta_a.
    \end{array}
  \end{equation*}
  For $f:\hom_A(a,b)$, we have
  \[ \inv{\eta_{b}}\circ G(F(f))\circ \eta_{a}  =
  \inv{\eta_{b}} \circ \eta_{b} \circ f=
  f
  \]
  while for $g:\hom_B(Fa,Fb)$ we have
  \begin{align*}
    F(\inv{\eta_b} \circ G(g)\circ\eta_a)
    &= F(\inv{\eta_b})\circ F(G(g))\circ F(\eta_a)\\
    &= \epsilon_{Fb}\circ F(G(g))\circ F(\eta_a)\\
    &= g\circ\epsilon_{Fa}\circ F(\eta_a)\\
    &= g
  \end{align*}
  using naturality of $\epsilon$, and the triangle identities twice.
  Thus, $F_{a,b}$ is an equivalence, so $F$ is fully faithful.
  Finally, for any $b:B$, we have $Gb:A$ and $\epsilon_b:FGb\cong b$.

  On the other hand, suppose $F$ is fully faithful and split essentially surjective.
  Define $G_0:B_0\to A_0$ by sending $b:B$ to the $a:A$ given by the specified essential splitting, and write $\epsilon_b$ for the likewise specified isomorphism $FGb\cong b$.

  Now for any $g:\hom_B(b,b')$, define $G(g):\hom_A(Gb,Gb')$ to be the unique morphism such that $\id{F(G(g))}{\inv{(\epsilon_{b'})}\circ g \circ \epsilon_b }$ (which exists since $F$ is fully faithful).
  Finally, for $a:A$ define $\eta_a:\hom_A(a,GFa)$ to be the unique morphism such that $\id{F\eta_a}{\inv{\epsilon_{Fa}}}$.
  It is easy to verify that $G$ is a functor and that $(G,\eta,\epsilon)$ exhibit $F$ as an equivalence of precategories.

  Now consider the composite~\ref{item:ct:ffeso1}$\to$\ref{item:ct:ffeso2}$\to$\ref{item:ct:ffeso1}.
  We clearly recover the same function $G_0:B_0 \to A_0$.
  For the action of $G$ on hom-sets, we must show that for $g:\hom_B(b,b')$, $G(g)$ is the (necessarily unique) morphism such that $F(G(g)) = \inv{(\epsilon_{b'})}\circ g \circ \epsilon_b$.
  But this equation holds by the assumed naturality of $\epsilon$.
  We also clearly recover $\epsilon$, while $\eta$ is uniquely characterized by $\id{F\eta_a}{\inv{\epsilon_{Fa}}}$ (which is one of the triangle identities assumed to hold in the structure of an equivalence of precategories).
  Thus, this composite is equal to the identity.

  Finally, consider the other composite~\ref{item:ct:ffeso2}$\to$\ref{item:ct:ffeso1}$\to$\ref{item:ct:ffeso2}.
  Since being fully faithful is a mere proposition, it suffices to observe that we recover, for each $b:B$, the same $a:A$ and isomorphism $F a \cong b$.
  But this is clear, since we used this function and isomorphism to define $G_0$ and $\epsilon$ in~\ref{item:ct:ffeso1}, which in turn are precisely what we used to recover~\ref{item:ct:ffeso2} again.
  Thus, the composites in both directions are equal to identities, hence we have an equivalence \eqv{\ref{item:ct:ffeso1}}{\ref{item:ct:ffeso2}}.
\end{proof}

However, if $B$ is not a category, then neither type in \autoref{ct:ffeso} may necessarily be a mere proposition.
This suggests considering as well the following notions.

\begin{defn}
  A functor $F:A\to B$ is \textbf{essentially surjective} if for all $b:B$, there \emph{merely} exists an $a:A$ such that $Fa\cong b$.
  We say $F$ is a \textbf{weak equivalence} if it is fully faithful and essentially surjective.
\end{defn}

Being a weak equivalence is \emph{always} a mere proposition.
For categories, however, there is no difference between equivalences and weak ones.

\begin{lem}\label{ct:catweq}
  If $F:A\to B$ is fully faithful and $A$ is a category, then for any $b:B$ the type $\sm{a:A} (Fa\cong b)$ is a mere proposition.
  Hence a functor between categories is an equivalence if and only if it is a weak equivalence.
\end{lem}
\begin{proof}
  Suppose given $(a,f)$ and $(a',f')$ in $\sm{a:A} (Fa\cong b)$.
  Then $\inv{f'}\circ f$ is an isomorphism $Fa \cong Fa'$.
  Since $F$ is fully faithful, we have $g:a\cong a'$ with $Fg = \inv{f'}\circ f$.
  And since $A$ is a category, we have $p:a=a'$ with $\idtoiso(p)=g$.
  Now $Fg = \inv{f'}\circ f$ implies $\trans{(\map{(F_0)}{p})}{f} = f'$, hence (by the characterization of equalities in dependent sums) $(a,f)=(a',f')$.

  Thus, for fully faithful functors whose domain is a category, essential surjectivity is equivalent to split essential surjectivity, and so being a weak equivalence is equivalent to being an equivalence.
\end{proof}

This is an important advantage of our category theory over set-based approaches.
With a purely set-based definition of category, the statement ``every fully faithful and essentially surjective functor is an equivalence of categories'' is equivalent to the mere axiom of choice (that is, the strong one, not the trivial one).
Here we have it for free, as a category-theoretic version of the function comprehension principle.
(In fact, this property characterizes categories among precategories; see \S\ref{sec:rezk}.)

On the other hand, the following characterization of equivalences of categories is perhaps even more useful.

\begin{defn}\label{ct:isocat}
  A functor $F:A\to B$ is an \textbf{isomorphism of (pre)categories} if $F$ is fully faithful and $F_0:A_0\to B_0$ is an equivalence of types.
\end{defn}

Note that being an isomorphism of precategories is always a mere property.
Let $A\cong B$ denote the type of isomorphisms of (pre)categories from $A$ to $B$.

\begin{lem}\label{ct:isoprecat}
  For precategories $A$ and $B$ and $F:A\to B$, the following are equivalent.
  \begin{enumerate}
  \item $F$ is an isomorphism of precategories.\label{item:ct:ipc1}
  \item There exist $G:B\to A$ and $\eta:1_A = GF$ and $\epsilon:FG=1_B$ such that\label{item:ct:ipc2}
    \begin{equation}
      \map{(\lambda H. F H)}{\eta} = \map{(\lambda K. K F)}{\opp\epsilon}.\label{eq:ct:isoprecattri}
    \end{equation}
  \item There merely exist $G:B\to A$ and $\eta:1_A = GF$ and $\epsilon:FG=1_B$.\label{item:ct:ipc3}
  \end{enumerate}
\end{lem}

Note that if $B_0$ is not a 1-type, then~\eqref{eq:ct:isoprecattri} may not be a mere proposition.

\begin{proof}
  First note that since hom-sets are sets, equalities between equalities of functors are uniquely determined by their object-parts.
  Thus, by function extensionality,~\eqref{eq:ct:isoprecattri} is equivalent to
  \begin{equation}
    \map{(F_0)}{\eta_0}_a = \opp{(\epsilon_0)}_{F_0 a}.\label{eq:ct:ipctri}
  \end{equation}
  for all $a:A_0$.
  Note that this is precisely the triangle identity for $G_0$, $\eta_0$, and $\epsilon_0$ to be a proof that $F_0$ is an equivalence of types.

  Now suppose~\ref{item:ct:ipc1}.
  Let $G_0:B_0 \to A_0$ be the inverse of $F_0$, with $\eta_0: \idfunc[A_0] = G_0 F_0$ and $\epsilon_0:F_0G_0 = \idfunc[B_0]$ satisfying the triangle identity, which is precisely~\eqref{eq:ct:ipctri}.
  Now define $G_{b,b'}:\hom_B(b,b') \to \hom_A(G_0b,G_0b')$ by
  \[ G_{b,b'}(g) \defeq
  \inv{(F_{G_0b,G_0b'})}\Big(\idtoiso(\opp{(\epsilon_0)}_{b'}) \circ g \circ \idtoiso((\epsilon_0)_b)\Big)
  \]
  (using the assumption that $F$ is fully faithful).
  Since \idtoiso takes opposites to inverses and concatenation to composition, and $F$ is a functor, it follows that $G$ is a functor.

  By definition, we have $(GF)_0 \jdeq G_0 F_0$, which is equal to $\idfunc[A_0]$ by $\eta_0$.
  To obtain $1_A = GF$, we need to show that when transported along $\eta_0$, the identity function of $\hom_A(a,a')$ becomes equal to the composite $G_{Fa,Fa'} \circ F_{a,a'}$.
  In other words, for any $f:\hom_A(a,a')$ we must have
  \begin{multline*}
    \idtoiso((\eta_0)_{a'}) \circ f \circ \idtoiso(\opp{(\eta_0)}_a)\\
    = \inv{(F_{GFa,GFa'})}\Big(\idtoiso(\opp{(\epsilon_0)}_{Fa'})
    \circ F_{a,a'}(f) \circ \idtoiso((\epsilon_0)_{Fa})\Big).
  \end{multline*}
  But this is equivalent to
  \begin{multline*}
    (F_{GFa,GFa'})\Big(\idtoiso((\eta_0)_{a'}) \circ f \circ \idtoiso(\opp{(\eta_0)}_a)\Big)\\
    = \idtoiso(\opp{(\epsilon_0)}_{Fa'})
    \circ F_{a,a'}(f) \circ \idtoiso((\epsilon_0)_{Fa}).
  \end{multline*}
  which follows from functoriality of $F$, the fact that $F$ preserves \idtoiso, and~\eqref{eq:ct:ipctri}.
  Thus we have $\eta:1_A = GF$.

  On the other side, we have $(FG)_0\jdeq F_0 G_0$, which is equal to $\idfunc[B_0]$ by $\epsilon_0$.
  To obtain $FG=1_B$, we need to show that when transported along $\epsilon_0$, the identity function of $\hom_B(b,b')$ becomes equal to the composite $F_{Gb,Gb'} \circ G_{b,b'}$.
  That is, for any $g:\hom_B(b,b')$ we must have
  \begin{multline*}
    F_{Gb,Gb'}\Big(\inv{(F_{Gb,Gb'})}\Big(\idtoiso(\opp{(\epsilon_0)}_{b'}) \circ g \circ \idtoiso((\epsilon_0)_b)\Big)\Big)\\
    = \idtoiso((\opp{\epsilon_0})_{b'}) \circ g \circ \idtoiso((\epsilon_0)_b).
  \end{multline*}
  But this is just the fact that $\inv{(F_{Gb,Gb'})}$ is the inverse of $F_{Gb,Gb'}$.
  And we have remarked that~\eqref{eq:ct:isoprecattri} is equivalent to~\eqref{eq:ct:ipctri}, so~\ref{item:ct:ipc2} holds.

  Conversely, suppose given~\ref{item:ct:ipc2}; then the object-parts of $G$, $\eta$, and $\epsilon$ together with~\eqref{eq:ct:ipctri} show that $F_0$ is an equivalence of types.
  And for $a,a':A_0$, we define $\overline{G}_{a,a'}: \hom_B(Fa,Fa') \to \hom_A(a,a')$ by
  \begin{equation}
    \overline{G}_{a,a'}(g) \defeq \idtoiso(\opp{\eta})_{a'} \circ G(g) \circ \idtoiso(\eta)_a.\label{eq:ct:gbar}
  \end{equation}
  By naturality of $\idtoiso(\eta)$, for any $f:\hom_A(a,a')$ we have
  \begin{align*}
    \overline{G}_{a,a'}(F_{a,a'}(f))
    &= \idtoiso(\opp{\eta})_{a'} \circ G(F(f)) \circ \idtoiso(\eta)_a\\
    &= \idtoiso(\opp{\eta})_{a'} \circ \idtoiso(\eta)_{a'} \circ f \\
    &= f.
  \end{align*}
  On the other hand, for $g:\hom_B(Fa,Fa')$ we have
  \begin{align*}
    F_{a,a'}(\overline{G}_{a,a'}(g))
    &= F(\idtoiso(\opp{\eta})_{a'}) \circ F(G(g)) \circ F(\idtoiso(\eta)_a)\\
    &= \idtoiso(\epsilon)_{Fa'}
    \circ F(G(g))
    \circ \idtoiso(\opp{\epsilon})_{Fa}\\
    &= \idtoiso(\epsilon)_{Fa'}
    \circ \idtoiso(\opp{\epsilon})_{Fa'}
    \circ g\\
    &= g.
  \end{align*}
  (There are lemmas needed here regarding the compatibility between \idtoiso and whiskering, which we leave it to the reader to state and prove.)
  Thus, $F_{a,a'}$ is an equivalence, so $F$ is fully faithful; i.e.~\ref{item:ct:ipc1} holds.

  Now the composite~\ref{item:ct:ipc1}$\to$\ref{item:ct:ipc2}$\to$\ref{item:ct:ipc1} is equal to the identity since~\ref{item:ct:ipc1} is a mere proposition.
  On the other side, tracing through the above constructions we see that the composite~\ref{item:ct:ipc2}$\to$\ref{item:ct:ipc1}$\to$\ref{item:ct:ipc2} essentially preserves the object-parts $G_0$, $\eta_0$, $\epsilon_0$, and the object-part of~\eqref{eq:ct:isoprecattri}.
  And in the latter three cases, the object-part is all there is, since hom-sets are sets.

  Thus, it suffices to show that we recover the action of $G$ on hom-sets.
  In other words, we must show that if $g:\hom_B(b,b')$, then
  \[ G_{b,b'}(g) =
  \overline{G}_{G_0b,G_0b'}\Big(\idtoiso(\opp{(\epsilon_0)}_{b'}) \circ g \circ \idtoiso((\epsilon_0)_b)\Big)
  \]
  where $\overline{G}$ is defined by~\eqref{eq:ct:gbar}.
  However, this follows from functoriality of $G$ and the \emph{other} triangle identity, which is equivalent to~\eqref{eq:ct:ipctri}.

  Now since~\ref{item:ct:ipc1} is a mere proposition, so is~\ref{item:ct:ipc2}, so it suffices to show they are co-inhabited with~\ref{item:ct:ipc3}.
  Of course,~\ref{item:ct:ipc2}$\to$\ref{item:ct:ipc3}, so let us assume~\ref{item:ct:ipc3}.
  Since~\ref{item:ct:ipc1} is a mere proposition, we may assume given $G$, $\eta$, and $\epsilon$.
  Then $G_0$ along with $\eta$ and $\epsilon$ imply that $F_0$ is an equivalence.
  Moreover, we also have natural isomorphisms $\idtoiso(\eta):1_A\cong GF$ and $\idtoiso(\epsilon):FG\cong 1_B$, so by \autoref{ct:adjointification}, $F$ is an equivalence of precategories, and in particular fully faithful.
\end{proof}

From \autoref{ct:isoprecat}\ref{item:ct:ipc2} and $\idtoiso$ in functor categories, we conclude immediately that any isomorphism of precategories is an equivalence.
For precategories, the converse can fail.

\begin{eg}\label{ct:chaotic}
  Let $X$ be a type and $x_0:X$ an element, and let $X_{\mathrm{ch}}$ denote the \emph{chaotic} or \emph{indiscrete} precategory on $X$.
  By definition, we have $(X_{\mathrm{ch}})_0\defeq X$, and $\hom_{X_{\mathrm{ch}}}(x,x') = 1$ for all $x,x'$.
  Then the unique functor $X_{\mathrm{ch}}\to 1$ is an equivalence of precategories, but not an isomorphism unless $X$ is contractible.

  This example also shows that a precategory can be equivalent to a category without itself being a category.
  Of course, if a precategory is \emph{isomorphic} to a category, then it must itself be a category.
\end{eg}

However, for categories, the two notions coincide.

\begin{lem}\label{ct:eqv-levelwise}
  For categories $A$ and $B$, a functor $F:A\to B$ is an equivalence of categories if and only if it is an isomorphism of categories.
\end{lem}
\begin{proof}
  Since both are mere properties, it suffices to show they are co-inhabited.
  So first suppose $F$ is an equivalence of categories, with $(G,\eta,\epsilon)$ given.
  We have already seen that $F$ is fully faithful.
  By \autoref{ct:functor-cat}, the natural isomorphisms $\eta$ and $\epsilon$ yield identities $\id{1_A}{GF}$ and $\id{FG}{1_B}$, hence in particular identities $\id{\idfunc[A]}{G_0\circ F_0}$ and $\id{F_0\circ G_0}{\idfunc[B]}$.
Thus, $F_0$ is an equivalence of types.

  Conversely, suppose $F$ is fully faithful and $F_0$ is an equivalence of types, with inverse $G_0$, say.
  Then for each $b:B$ we have $G_0 b:A$ and an identity $\id{FGb}{b}$, hence an isomorphism $FGb\cong b$.
  Thus, by \autoref{ct:ffeso}, $F$ is an equivalence of categories.
\end{proof}

Of course, there is yet a third notion of sameness for (pre)categories: equality.
However, the univalence axiom implies that it coincides with isomorphism.

\begin{lem}\label{ct:cat-eq-iso}
  If $A$ and $B$ are precategories, then the function
  \[(\id A B) \to (A\cong B)\]
  (defined by induction from the identity functor) is an equivalence of types.
\end{lem}
\begin{proof}
  As usual for dependent sum types, to give an element of $\id A B$ is equivalent to giving
  \begin{itemize}
  \item an identity $P_0:\id{A_0}{B_0}$,
  \item for each $a,b:A_0$, an identity
    \[P_{a,b}:\id{\hom_A(a,b)}{\hom_B(\trans {P_0} a,\trans {P_0} b)},\]
  \item identities $\id{\trans {(P_{a,a})} {1_a}}{1_{\trans {P_0} a}}$ and $\id{\trans {(P_{a,c})} {gf}}{\trans {(P_{b,c})} g \circ \trans {(P_{a,b})} f}$.
  \end{itemize}
  (Again, we use the fact that the identity types of hom-sets are mere propositions.)
  However, by univalence, this is equivalent to giving
  \begin{itemize}
  \item an equivalence of types $F_0:\eqv{A_0}{B_0}$,
  \item for each $a,b:A_0$, an equivalence of types
    \[F_{a,b}:\eqv{\hom_A(a,b)}{\hom_B(F_0 (a),F_0 (b))},\]
  \item and identities $\id{F_{a,a}(1_a)}{1_{F_0 (a)}}$ and $\id{F_{a,c}(gf)}{F_{b,c} (g)\circ F_{a,b} (f)}$.
  \end{itemize}
  But this consists exactly of a functor $F:A\to B$ that is an isomorphism of categories.
  And by induction on identity, this equivalence $\eqv{(\id A B)}{(A\cong B)}$ is equal to the one obtained by induction.
\end{proof}

Thus, for categories, equality also coincides with equivalence.
We can interpret this as saying that categories, functors, and natural transformations form, not just a pre-2-category, but a 2-category.

\begin{thm}\label{ct:cat-2cat}
  If $A$ and $B$ are categories, then the function
  \[(\id A B) \to (A\simeq B)\]
  (defined by induction from the identity functor) is an equivalence of types.
\end{thm}
\begin{proof}
  By \autoref{ct:cat-eq-iso} and \autoref{ct:eqv-levelwise}.
\end{proof}

As a consequence, the type of categories is a 2-type.
For since $A\simeq B$ is a subtype of the type of functors from $A$ to $B$, which are the objects of a category, it is a 1-type; hence the identity types $\id A B$ are also 1-types.


\section{The Yoneda lemma}
\label{sec:yoneda}

Recall that we have a category \uset whose objects are sets and whose morphisms are functions.
We now show that every precategory has a \uset-valued hom-functor.
First we need to define opposites and products of (pre)categories.

\begin{defn}
  For a precategory $A$, its \textbf{opposite} $A\op$ is a precategory with the same type of objects, with $\hom_{A\op}(a,b) \defeq \hom_A(b,a)$, and with identities and composition inherited from $A$.
\end{defn}

\begin{defn}
  For precategories $A$ and $B$, their \textbf{product} $A\times B$ is a precategory with $(A\times B)_0 \defeq A_0 \times B_0$ and
  \[\hom_{A\times B}((a,b),(a',b')) \defeq \hom_A(a,a') \times \hom_B(b,b').\]
  Identities are defined by $1_{(a,b)}\defeq (1_a,1_b)$ and composition by $(g,g')(f,f') \defeq ((gf),(g'f'))$.
\end{defn}

\begin{lem}\label{ct:functorexpadj}
  For precategories $A,B,C$, the following types are equivalent.
  \begin{enumerate}
  \item Functors $A\times B\to C$.
  \item Functors $A\to C^B$.
  \end{enumerate}
\end{lem}
\begin{proof}
  Given $F:A\times B\to C$, for any $a:A$ we obviously have a functor $F_a : B\to C$.
  This gives a function $A_0 \to (C^B)_0$.
  Next, for any $f:\hom_A(a,a')$, we have for any $b:B$ the morphism $F_{(a,b),(a',b)}(f,1_b):F_a(b) \to F_{a'}(b)$.
  These are the components of a natural transformation $F_a \to F_{a'}$.
  Functoriality in $a$ is easy to check, so we have a functor $\hat{F}:A\to C^B$.

  Conversely, suppose given $G:A\to C^B$.
  Then for any $a:A$ and $b:B$ we have the object $G(a)(b):C$, giving a function $A_0 \times B_0 \to C_0$.
  And for $f:\hom_A(a,a')$ and $g:\hom_B(b,b')$, we have the morphism
  \begin{equation*}
     G(a')_{b,b'}(g)\circ G_{a,a'}(f)_b = G_{a,a'}(f)_{b'} \circ  G(a)_{b,b'}(g)
  \end{equation*}
  in $\hom_C(G(a)(b), G(a')(b'))$.
  Functoriality is again easy to check, so we have a functor $\check{F}:A\times B \to C$.

  Finally, it is also clear that these operations are inverses.
\end{proof}

Now for any precategory $A$, we have a hom-functor
\[\hom_A : A\op \times A \to \uset.\]
It takes a pair $(a,b): (A\op)_0 \times A_0 \jdeq A_0 \times A_0$ to the set $\hom_A(a,b)$.
For a morphism $(f,f') : \hom_{A\op\times A}((a,b),(a',b'))$, by definition we have $f:\hom_A(a',a)$ and $f':\hom_A(b,b')$, so we can define
\begin{align*}
  (\hom_A)_{(a,b),(a',b')}(f,f')
  &\defeq (g \mapsto (f'gf))\\
  &: \hom_A(a,b) \to \hom_A(a',b').
\end{align*}
Functoriality is easy to check.

By \autoref{ct:functorexpadj}, therefore, we have an induced functor $\y:A\to \uset^{A\op}$, which we call the \textbf{Yoneda embedding}.

\begin{thm}[The Yoneda lemma]\label{ct:yoneda}
  For any precategory $A$, any $a:A$, and any functor $F:\uset^{A\op}$, we have an isomorphism
  \begin{equation}\label{eq:yoneda}
    \hom_{\uset^{A\op}}(\y a, F) \cong Fa.
  \end{equation}
  Moreover, this is natural in both $a$ and $F$.
\end{thm}
\begin{proof}
  Given a natural transformation $\alpha:\y a \to F$, we can consider the component $\alpha_a : \y a(a) \to F a$.
  Since $\y a(a)\jdeq \hom_A(a,a)$, we have $1_a : \y a(a)$, so that $\alpha_a(1_a) : F a$.
  This gives a function $(\alpha \mapsto \alpha_a(1_a))$ from left to right in~\eqref{eq:yoneda}.

  In the other direction, given $x:F a$, we define $\alpha:\y a \to F$ by
  \[\alpha_{a'}(f) \defeq F_{a',a}(f)(x). \]
  Naturality is easy to check, so this gives a function from right to left in~\eqref{eq:yoneda}.

  To show that these are inverses, first suppose given $x:F a$.
  Then with $\alpha$ defined as above, we have $\alpha_a(1_a) = F_{a,a}(1_a)(x) = 1_{F a}(x) = x$.
  On the other hand, if we suppose given $\alpha:\y a \to F$ and define $x$ as above, then for any $f:\hom_A(a',a)$ we have
  \begin{align*}
    \alpha_{a'}(f)
    &= \alpha_{a'} (\y a_{a',a}(f))\\
    &= (\alpha_{a'}\circ \y a_{a',a}(f))(1_a)\\
    &= (F_{a',a}(f)\circ \alpha_a)(1_a)\\
    &= F_{a',a}(f)(\alpha_a(1_a))\\
    &= F_{a',a}(f)(x).
  \end{align*}
  Thus, both composites are equal to identities.
  We leave the proof of naturality to the reader.
\end{proof}

\begin{cor}\label{ct:yoneda-embedding}
  The Yoneda embedding $\y :A\to \uset^{A\op}$ is fully faithful.
\end{cor}
\begin{proof}
  By \autoref{ct:yoneda}, we have
  \[ \hom_{\uset^{A\op}}(\y a, \y b) \cong \y b(a) \jdeq \hom_A(a,b). \]
  It is easy to check that this isomorphism is in fact the action of \y on hom-sets.
\end{proof}

\begin{cor}\label{ct:yoneda-mono}
  If $A$ is a category, then $\y_0 : A_0 \to (\uset^{A\op})_0$ is a monomorphism.
  In particular, if $\y a = \y b$, then $a=b$.
\end{cor}
\begin{proof}
  By \autoref{ct:yoneda-embedding}, \y induces an isomorphism on sets of isomorphisms.
  But as $A$ and $\uset^{A\op}$ are categories and \y is a functor, this is equivalently an isomorphism on identity types, which is the definition of being mono.
\end{proof}

\begin{defn}\label{ct:representable}
  A functor $F:\uset^{A\op}$ is said to be \textbf{representable} if there exists $a:A$ and an isomorphism $\y a \cong F$.
\end{defn}

\begin{thm}\label{ct:representable-prop}
  If $A$ is a category, then the type ``$F$ is representable'' is a mere proposition.
\end{thm}
\begin{proof}
  By definition ``$F$ is representable'' is just the fiber of $\y_0$ over $F$.
  Since $\y_0$ is mono by \autoref{ct:yoneda-mono}, this fiber is a mere proposition.
\end{proof}

In particular, in a category, any two representations of the same functor are equal.
We could use this to give a different proof of \autoref{ct:adjprop} by characterizing adjunctions in terms of representability.


\section{The Rezk completion}
\label{sec:rezk}

In this section we will give a universal way to replace a precategory by a category.
It relies on the fact that ``categories see weak equivalences as equivalences''.

To prove this, we begin with a couple of lemmas which are completely standard category theory, phrased carefully so as to make sure we are using the eliminator for $\truncf{-1}$ correctly.
One would have to be similarly careful in classical category theory if one wanted to avoid the axiom of choice: any time we want to define a function, we need to characterize its values uniquely somehow.

\begin{lem}\label{lem:precomp-faithful}
  If $A,B,C$ are precategories and $H:A\to B$ is an essentially surjective functor, then $(-\circ H):C^B \to C^A$ is faithful.
\end{lem}
\begin{proof}
  Let $F,G:B\to C$, and $\gamma,\delta:F\to G$ be such that $\gamma H = \delta H$; we must show $\gamma=\delta$.
  Thus let $b:B$; we want to show $\gamma_b=\delta_b$.
  This is a mere proposition, so since $H$ is essentially surjective, we may assume given an $a:A$ and an isomorphism $f:Ha\cong b$.
  But now we have
  \[ \gamma_b = G(f) \circ \gamma_{Ha} \circ F(\inv{f})
  = G(f) \circ \delta_{Ha} \circ F(\inv{f})
  = \delta_b.\qedhere
  \]
\end{proof}

\begin{lem}\label{lem:precomp-full}
  If $A,B,C$ are precategories and $H:A\to B$ is essentially surjective and full, then $(-\circ H):C^B \to C^A$ is fully faithful.
\end{lem}
\begin{proof}
  It remains to show fullness.
  Thus, let $F,G:B\to C$ and $\gamma:FH \to GH$.
  We claim that for any $b:B$, the type
  \begin{equation}\label{eq:fullprop}
    \sm{g:\hom_C(Fb,Gb)} \prd{a:A}{f:Ha\cong b} (\gamma_a =  \inv{Gf}\circ g\circ Ff)
  \end{equation}
  is contractible.
  Since contractibility is a mere property, and $H$ is essentially surjective, we may assume given $a_0:A$ and $h:Ha_0\cong b$.

  Now take $g\defeq Gh \circ \gamma_{a_0} \circ \inv{Fh}$.
  Then given any other $a:A$ and $f:Ha\cong b$, we must show $\gamma_a =  \inv{Gf}\circ g\circ Ff$.
  Since $H$ is full, there merely exists a morphism $k:\hom_A(a,a_0)$ such that $Hk = \inv{h}\circ f$.
  And since our goal is a mere proposition, we may assume given some such $k$.
  Then we have
  \begin{align*}
    \gamma_a &= \inv{GHk}\circ \gamma_{a_0} \circ FHk\\
    &= \inv{Gf} \circ Gh \circ \gamma_{a_0} \circ \inv{Fh} \circ Ff\\
    &= \inv{Gf}\circ g\circ Ff.
  \end{align*}
  Thus,~\eqref{eq:fullprop} is inhabited.
  It remains to show it is a mere proposition.
  Let $g,g':\hom_C(Fb, Gb)$ be such that for all $a:A$ and $f:Ha\cong b$, we have both $(\gamma_a =  \inv{Gf}\circ g\circ Ff)$ and $(\gamma_a =  \inv{Gf}\circ g'\circ Ff)$.
  The dependent product types are mere propositions, so all we have to prove is $g=g'$.
  But this is a mere proposition, so we may assume $a_0:A$ and $h:Ha_0\cong b$, in which case we have
  \[ g = Gh \circ \gamma_{a_0} \circ \inv{Fh} = g'.\]

  This proves that~\eqref{eq:fullprop} is contractible for all $b:B$.
  Now we define $\delta:F\to G$ by taking $\delta_b$ to be the unique $g$ in~\eqref{eq:fullprop} for that $b$.
  To see that this is natural, suppose given $f:\hom_B(b,b')$; we must show $Gf \circ \delta_b = \delta_{b'}\circ Ff$.
  As before, we may assume $a:A$ and $h:Ha\cong b$, and likewise $a':A$ and $h':Ha'\cong b'$.
  Since $H$ is full as well as essentially surjective, we may also assume $k:\hom_A(a,a')$ with $Hk = \inv{h'}\circ f\circ h$.

  Since $\gamma$ is natural, $GHk\circ \gamma_a = \gamma_{a'} \circ FHk$.
  Using the definition of $\delta$, we have
  \begin{align*}
    Gf \circ \delta_b
    &= Gf \circ Gh \circ \gamma_a \circ \inv{Fh}\\
    &= Gh' \circ GHk\circ \gamma_a \circ \inv{Fh}\\
    &= Gh' \circ \gamma_{a'} \circ FHk \circ \inv{Fh}\\
    &= Gh' \circ \gamma_{a'} \circ \inv{Fh'} \circ Ff\\
    &= \delta_{b'} \circ Ff.
  \end{align*}
  Thus, $\delta$ is natural.
  Finally, for any $a:A$, applying the definition of $\delta_{Ha}$ to $a$ and $1_a$, we obtain $\gamma_a = \delta_{Ha}$.
  Hence, $\delta \circ H = \gamma$.
\end{proof}

The rest of the theorem follows almost exactly the same lines, with the category-ness of $C$ inserted in one crucial step, which we have bolded below for emphasis.
This is the point at which we are trying to define a function into \emph{objects} without using choice, and so we must be careful about what it means for an object to be ``uniquely specified''.
In classical category theory, all one can say is that this object is specified up to unique isomorphism, but in set-theoretic foundations this is not a sufficient amount of uniqueness to give us a function without invoking AC.
In univalent foundations, however, if $C$ is a category, then isomorphism is equality, and we have the appropriate sort of uniqueness (namely, living in a contractible space).

\begin{thm}\label{ct:cat-weq-eq}
  If $A,B$ are precategories, $C$ is a category, and $H:A\to B$ is a weak equivalence, then $(-\circ H):C^B \to C^A$ is an isomorphism.
\end{thm}
\begin{proof}
  By \autoref{ct:functor-cat}, $C^B$ and $C^A$ are categories.
  Thus, by \autoref{ct:eqv-levelwise} it will suffice to show that $(-\circ H)$ is an equivalence.
  But since we know from the preceeding two lemmas that it is fully faithful, by \autoref{ct:catweq} it will suffice to show that it is essentially surjective.
  Thus, suppose $F:A\to C$; we want there to merely exist a $G:B\to C$ such that $GH\cong F$.

  For each $b:B$, let $X_b$ be the type whose elements consist of:
  \begin{enumerate}
  \item An element $c:C$; and
  \item For each $a:A$ and $h:Ha\cong b$, an isomorphism $k_{a,h}:Fa\cong c$; such that\label{item:eqvprop2}
  \item For each $(a,h)$ and $(a',h')$ as in~\ref{item:eqvprop2} and each $f:\hom_A(a,a')$ such that $h'\circ Hf = h$, we have $k_{a',h'}\circ Ff = k_{a,h}$.\label{item:eqvprop3}
  \end{enumerate}
  We claim that for any $b:B$, the type $X_b$ is contractible.
  As this is a mere proposition, we may assume given $a_0:A$ and $h_0:Ha_0 \cong b$.
  Let $c^0\defeq Fa_0$.
  Next, given $a:A$ and $h:Ha\cong b$, since $H$ is fully faithful there is a unique isomorphism $g_{a,h}:a\to a_0$ with $Hg_{a,h} = \inv{h_0}\circ h$; define $k^0_{a,h} \defeq Fg_{a,h}$.
  Finally, if $h'\circ Hf = h$, then $\inv{h_0}\circ h'\circ Hf = \inv{h_0}\circ h$, hence $g_{a',h'} \circ f = g_{a,h}$ and thus $k^0_{a',h'}\circ Ff = k^0_{a,h}$.
  Therefore, $X_b$ is inhabited.

  Now suppose given another $(c^1,k^1): X_b$.
  Then $k^1_{a_0,h_0}:c^0 \jdeq Fa_0 \cong c^1$.
  \textbf{Since $C$ is a category, we have $p:c^0=c^1$ with $\idtoiso(p) = k^1_{a_0,h_0}$.}
  And for any $a:A$ and $h:Ha\cong b$, by~\ref{item:eqvprop3} for $(c^1,k^1)$ with $f\defeq g_{a,h}$, we have
  \[k^1_{a,h} = k^1_{a_0,h_0} \circ k^0_{a,h} = \trans{p}{k^0_{a,h}}\]
  This gives the requisite data for an equality $(c^0,k^0)=(c^1,k^1)$, completing the proof that $X_b$ is contractible.

  Now since $X_b$ is contractible for each $b$, the type $\prd{b:B} X_b$ is also contractible.
  In particular, it is inhabited, so we have a function assigning to each $b:B$ a $c$ and a $k$.
  Define $G_0(b)$ to be this $c$; this gives a function $G_0 :B_0 \to C_0$.

  Next we need to define the action of $G$ on morphisms.
  For each $b,b':B$ and $f:\hom_B(b,b')$, let $Y_f$ be the type whose elements consist of:
  \begin{enumerate}[resume]
  \item A morphism $g:\hom_C(Gb,Gb')$, such that
  \item For each $a:A$ and $h:Ha\cong b$, and each $a':A$ and $h':Ha'\cong b'$, and any $\ell:\hom_A(a,a')$, we have\label{item:eqvprop5}
    \[ (h' \circ H\ell = f \circ h)
    \to
    (k_{a',h'} \circ F\ell = g\circ k_{a,h}). \]
  \end{enumerate}
  We claim that for any $b,b'$ and $f$, the type $Y_f$ is contractible.
  As this is a mere proposition, we may assume given $a_0:A$ and $h_0:Ha_0\cong b$, and each $a'_0:A$ and $h'_0:Ha'_0\cong b'$.
  Then since $H$ is fully faithful, there is a unique $\ell_0:\hom_A(a_0,a_0')$ such that $h'_0 \circ H\ell_0 = f \circ h_0$.
  Define $g_0 \defeq k_{a_0',h_0'} \circ F \ell_0 \circ \inv{(k_{a_0,h_0})}$.

  Now for any $a,h,a',h'$, and $\ell$ such that $(h' \circ H\ell = f \circ h)$, we have $\inv{h}\circ h_0:Ha_0\cong Ha$, hence there is a unique $m:a_0\cong a$ with $Hm = \inv{h}\circ h_0$ and hence $h\circ Hm = h_0$.
  Similarly, we have a unique $m':a_0'\cong a'$ with $h'\circ Hm' = h_0'$.
  Now by~\ref{item:eqvprop3}, we have $k_{a,h}\circ Fm = k_{a_0,h_0}$ and $k_{a',h'}\circ Fm' = k_{a_0',h_0'}$.
  We also have
  \begin{align*}
    Hm' \circ H\ell_0
    &= \inv{(h')} \circ h_0' \circ H\ell_0\\
    &= \inv{(h')} \circ f \circ h_0\\
    &= \inv{(h')} \circ f \circ h \circ \inv{h} \circ h_0\\
    &= H\ell \circ Hm
  \end{align*}
  and hence $m'\circ \ell_0 = \ell\circ m$ since $H$ is fully faithful.
  Finally, we can compute
  \begin{align*}
    g_0 \circ k_{a,h}
    &= k_{a_0',h_0'} \circ F \ell_0 \circ \inv{(k_{a_0,h_0})} \circ k_{a,h}\\
    &= k_{a_0',h_0'} \circ F \ell_0 \circ \inv{Fm}\\
    &= k_{a_0',h_0'} \circ \inv{(Fm')} \circ F\ell\\
    &= k_{a',h'}\circ F\ell.
  \end{align*}
  Whew!  We've shown that $Y_f$ is inhabited.
  To show it is contractible, since hom-sets are sets, it thankfully suffices to take another $g_1:\hom_C(Gb,Gb')$ satisfying~\ref{item:eqvprop5} and show $g_0=g_1$.
  However, we still have our specified $a_0,h_0,a_0',h_0',\ell_0$ around, and~\ref{item:eqvprop5} implies both $g_0$ and $g_1$ must be equal to $k_{a_0',h_0'} \circ F \ell_0 \circ \inv{(k_{a_0,h_0})}$.

  This completes the proof that $Y_f$ is contractible for each $b,b':B$ and $f:\hom_B(b,b')$.
  Therefore, there is a function assigning to each such $f$ its unique inhabitant; denote this function $G_{b,b'}:\hom_B(b,b') \to \hom_C(Gb,Gb')$.
  The proof that $G$ is a functor is straightforward; in each case we can choose $a,h$ and apply~\ref{item:eqvprop5}.

  Finally, for any $a_0:A$, defining $c\defeq Fa_0$ and $k_{a,h}\defeq F g$, where $g:\hom_A(a,a_0)$ is the unique isomorphism with $Hg = h$, gives an element of $X_{Ha_0}$.
  Thus, it is equal to the specified one; hence $GHa=Fa$.
  Similarly, for $f:\hom_A(a_0,a_0')$ we can define an element of $Y_{Hf}$ by transporting along these equalities, which must therefore be equal to the specified one.
  Hence, we have $GH=F$, and thus $GH\cong F$ as desired.
\end{proof}

Therefore, if a precategory $A$ admits a weak equivalence functor $A\to \hat{A}$, then that is its ``reflection'' into categories: any functor from $A$ into a category will factor essentially uniquely through $\widehat{A}$.
We now construct such a weak equivalence.

\begin{thm}\label{thm:rezk-completion}
  For any precategory $A$, there is a category $\widehat A$ and a weak equivalence $A\to\widehat{A}$.
\end{thm}
\begin{proof}
  Let $\widehat{A}_0 \defeq \setof{ F:\uset^{A\op} | \Brck{\sm{a:A} (\y a \cong F)}}$, with hom-sets inherited from $\uset^{A\op}$.
  Then the inclusion $\widehat{A} \to \uset^{A\op}$ is fully faithful and a monomorphism on objects.
  Since $\uset^{A\op}$ is a category (by \autoref{ct:functor-cat}, since \uset is so by univalence), $\widehat A$ is also a category.

  Let $A\to\widehat A$ be the Yoneda embedding.
  This is fully faithful by \autoref{ct:yoneda-embedding}, and essentially surjective by definition of $\widehat{A}_0$.
  Thus it is a weak equivalence.
\end{proof}

\begin{rmk}
  Our proof does have the drawback that it increases universe level.
  Namely, if $A$ is a category in a universe \bbU, then in this proof \uset must be at least as large as $\uset_\bbU$.
  Then $\uset_\bbU$ and $(\uset_\bbU)^{A\op}$ are not themselves categories in \bbU, but only in a higher universe, and \emph{a priori} the same is true of $\widehat A$.
  One could imagine a resizing axiom that could deal with this, but it is also possible to give a direct construction using higher inductive types.
\end{rmk}

We call the construction $A\mapsto \widehat A$ the \textbf{Rezk completion}, although there is also an argument (coming from higher topos semantics) for calling it the \textbf{stack completion}.

We have seen that most precategories arising in practice are categories, since they are constructed from \uset, which is a category by the univalence axiom.
However, there are a few cases in which the Rezk completion is necessary to obtain a category.

\begin{eg}
  Recall from \autoref{ct:fundgpd} that for any type $X$ there is a pregroupoid with $X$ as its type of objects and $\hom(x,y) \defeq \pizero{x=y}$.
  Its Rezk completion is the \emph{fundamental groupoid} of $X$.
  Recalling that groupoids are equivalent to 1-types, it is not hard to identify this groupoid with $\trunc1X$.
\end{eg}

\begin{eg}\label{ct:hocat}
  Recall from \autoref{ct:hoprecat} that there is a precategory whose type of objects is \type and with $\hom(X,Y) \defeq \pizero{X\to Y}$.
  Its Rezk completion may be called the \emph{homotopy category of types}.
  Its type of objects can be identified with $\trunc1\type$.
\end{eg}

The Rezk completion also allows us to show that the notion of ``category'' is determined by the notion of ``weak equivalence of precategories''.
Thus, insofar as the latter is inevitable, so is the former.

\begin{thm}
  A precategory $C$ is a category if and only if for every weak equivalence of precategories $H:A\to B$, the induced functor $(-\circ H):C^B \to C^A$ is an isomorphism of precategories.
\end{thm}
\begin{proof}
  ``Only if'' is \autoref{ct:cat-weq-eq}.
  In the other direction, let $H$ be $I:A\to\widehat A$.
  Then since $(-\circ I)_0$ is an equivalence, there exists $R:\widehat A\to A$ such that $RI=1_A$.
  Hence $IRI=I$, but again since $(-\circ I)_0$ is an equivalence, this implies $IR =1_{\widehat A}$.
  By \autoref{ct:isoprecat}\ref{item:ct:ipc3}, $I$ is an isomorphism of precategories.
  But then since $\widehat A$ is a category, so is $A$.
\end{proof}



\section{The Formalization}
\label{sec:formalization}
% \newcommand{\coq}{\textsf{Coq}}   % this doesn't work properly

Large chunks of the material presented above have been formalized in the proof assistant \textsf{Coq}.
The version of \textsf{Coq} used is \textsf{Coq} 8.3pl5, patched according to the instructions given by
V.\ Voevodsky\footnote{\url{https://github.com/vladimirias/Foundations/tree/master/Coq_patches}}.

Our formalization is based on Voevodsky's \emph{Foundations} library \cite{vv_foundations},
and is available online\footnote{\url{https://github.com/benediktahrens/rezk_completion}}.


\subsection*{Design principles}
Our general design principles largely follow the conventions established by Voevodsky in the library \cite{vv_foundations} 
with a few departures. 
Both use only three type constructors: $\Pi$, $\Sigma$, $\textsf{Id}$ and do not use 
almost any of the syntactic sugar of \textsf{Coq} such as the record types. 
Following Voevodsky, we use implicit arguments and, quite extensively, coercions. 

We restrict ourselves to these basic type constructors since they have a well-understood
semantics in various homotopy-theoretic models. Implicit arguments and coercions are 
crucial to manage structures of high complexity. Furthermore, they reflect
 familiar mathematical practice.


As for the differences, the use of notations, especially with infix symbols (for example, \lstinline!f ;; g! for 
the composition of morphisms of a precategory) plays an important role in our formalization. 
We also use the section mechanism of \textsf{Coq} 
when several hypotheses are common to a series of constructions and lemmas, e.g., 
when constructing particular examples of complex structures.

\subsection*{Reading the code}

Since informal type theory, used in the previous sections, is supposed to match its formal equivalent quite closely, 
the statements of the formalization are very similar to the corresponding statements of the informal type theory. 
For example, our formal statement correponding to \autoref{def:whisker} looks as follows:
\begin{lstlisting}
Lemma is_nat_trans_pre_whisker (A B C: precategory) (F: functor A B)
   (G H: functor B C) (gamma: nat_trans G H) :
  is_nat_trans (G o F) (H o F) (fun a: ob A => gamma (F a)).
\end{lstlisting}


The major differences occur when we split a large definition in parts as, for example, for the definition of a precategory. 
We first, define:


\begin{lstlisting}
Definition precategory_ob_mor := total2 (
  fun ob : UU => ob -> ob -> hSet).
\end{lstlisting}
%
Given an element \lstinline!C! of the above type, we write \lstinline!ob C!
for its first component and \lstinline!a --> b! for the value of the second 
component on \lstinline!a b : ob C!.

We complete the data of a precategory by:

\begin{lstlisting}
Definition precategory_data := total2 (
   fun C : precategory_ob_mor =>
     dirprod (forall c : ob C, c --> c)
             (forall a b c : ob C, a --> b -> b --> c -> a --> c)).
\end{lstlisting}
In the following we write \lstinline!identity c! for the identity morphism
on an object \lstinline!c! and \lstinline!f ;; g! for the composite of 
morphisms \lstinline!f : a --> b! and \lstinline!g : b --> c!.

We define a predicate expressing that this data constitutes a precategory:
\begin{lstlisting}
Definition is_precategory (C : precategory_data) :=
   dirprod (dirprod (forall (a b : ob C) (f : a --> b),
                         identity a ;; f == f)
                     (forall (a b : ob C) (f : a --> b),
                         f ;; identity b == f))
            (forall (a b c d : ob C)
                    (f : a --> b)(g : b --> c) (h : c --> d),
                     f ;; (g ;; h) == (f ;; g) ;; h).
\end{lstlisting}
As the last step, we say that a precategory is given by the data of a precategory
satisfying the necessary axioms:
\begin{lstlisting}
Definition precategory := total2 is_precategory.
\end{lstlisting}







\subsection*{Contents of the formalization}

In this part of the project we aimed on formalizing the Rezk completion together with its universal property. The formalization consists of 10 files:
\begin{itemize}
 \item \texttt{precategories.v} which roughly covers \autoref{sec:cats}.
 \item \texttt{functors\textunderscore transformations.v} which roughly covers \autoref{sec:transfors}.
 \item \texttt{sub\textunderscore precategories.v} where we define sub-precategories and
                 the image factorization of a functor. 
          This is not a separate part of the paper, but it is used (in less generality) in \autoref{thm:rezk-completion}.
 \item \texttt{equivalences.v} where we cover parts of \autoref{sec:equivalences} needed for \autoref{ct:cat-weq-eq}.
 \item \texttt{category\textunderscore hset.v} where we define the precategory of sets and show that it is a category.
 \item \texttt{yoneda.v} where we cover the main parts of \autoref{ct:yoneda}.
 \item \texttt{whiskering.v} where we define the whiskering, see \autoref{def:whisker}.
 \item \texttt{precomp\textunderscore fully\textunderscore faithful.v} that covers \autoref{lem:precomp-faithful} and \ref{lem:precomp-full}.
 \item \texttt{precomp\textunderscore ess\textunderscore surj.v} that covers \autoref{ct:cat-weq-eq}.
 \item \texttt{Rezk\textunderscore completion.v} that puts the previous files together exhibiting \autoref{thm:rezk-completion}.
\end{itemize}




%\subsubsection*{Issues}
%Our library has an unsatisfactory aspect due to the design principles outlined above.
%Packaging of structures---i.e., data types together with operations and properties of these operations---in 
% $\Sigma$-types often requires proving similar statements several times. For example, we have to define the composition of
% isomorphisms on top of---and in terms of---already existing composition of morphisms in a precategory and then prove 
% some of its properties that have already been proven---or even given as axioms---for their underlying morphisms.
% Using techniques such as type classes, facts of this kind could be established once and for all and various properties 
% of isomorphisms (such as stability under composition) could be inferred automatically by the system whenever needed.


\subsection*{Formalization vs informal definitions}

The formalization deviates very little from the informal definitions given in the previous sections.
We shall mention here the only example of such a deviation, resulting in a slicker definition. 
In \autoref{def:adjoint} the natural transformations $(\epsilon F)$ and $(F\eta)$ (similarly, 
$(G\epsilon)$ and $(\eta G)$) are actually not
composable! We have $\epsilon F : (FG)F \to 1_{B}F$ and $F\eta : F1_A \to F(GF)$. 
However, $(FG)F$ and $F(GF)$ are not convertible in our system,
which is necessary for the composition to typecheck. So in order to state the equality in question
we would have to insert extra coherence data.

We overcome this issue by rephrasing the axiom: instead of requiring an equality of natural 
transformations, we require it to hold pointwise. 
These statements are logically and type-theoretically equivalent, but for the latter we have the
desired convertibility. 
Indeed, for any $a : A$, we have $\big(F(GF)\big)(a)$ is convertible to $\big((FG)F\big)(a)$.



\subsection*{Statistics}

Our library comprises ten files with ca.\ 180 definitions and 170 lemmas altogether.
The \texttt{coqwc} tool counts 1200 lines of specification --- definitions and statements of lemmas and theorems ---
and 2700 lines of proof script overall.




%%% Local Variables:
%%% mode: latex
%%% TeX-master: "hottcats"
%%% End:


\section{Categorical semantics}
\label{sec:semantics}

While type theory can fruitfully be regarded as a foundational system for mathematics in its own right, part of its power is that it can also be interpreted internally in many different categories, such as toposes.
Thus, a proof performed in type theory is valid internally in any such category.
This can greatly simplify some proofs, enabling us to use a familiar language of types, elements, and equality, rather than having to manually translate them into objects, arrows, and commutative diagrams.
In essence, the general machine of categorical semantics functions like a ``compiler'' which automatically translates the former language into the latter.

For homotopy type theory, the relevant categorical semantics takes place in \emph{higher} categories.
The general form of this semantics is not fully worked out, but a good amount is known, especially in the 1-truncated situation with which we have been mostly concerned in this paper.
In this section we will describe the higher-categorical semantics of our type-theoretic notions of precategory and category.

\subsection{Groupoids}
\label{sec:groupoids}

The first higher-categorical model of type theory was the groupoid model of~\cite{hs:gpd-typethy}.
In this model:
\begin{itemize}
\item A type $A$ (or more generally a context) is represented by a small groupoid $\m A$.
\item A dependent type $\Gamma\vdash A$ is represented by a functor $\m A :\m \Gamma \to \mathrm{Gpd}$, where $\mathrm{Gpd}$ is the (large) groupoid of small groupoids and isomorphisms between them.
\item Given this, the context extension $\Gamma,(x:A)$ is represented by the ``Grothendieck construction'' of the functor $\m A$.
  Its objects are pairs $(x,y)$ where $x\in\m\Gamma$ and $y\in \m A(x)$, and its morphisms $(x,y)\to(x',y')$ are pairs $(f,g)$ where $f:x\to x'$ and $g:\m A(f)(y) \to y'$.
  The forgetful functor from this groupoid to $\m\Gamma$ is a split fibration, and every split fibration arises up to isomorphism in this way.
  Thus, categorically speaking, we may also identify dependent types with split fibrations.
\item A term $\Gamma \vdash (t:A)$ in a dependent type is represented by a section of the corresponding split fibration, assigning to each $x\in\m\Gamma$ an object $\m t(x)\in \m A(x)$, and to each $f:x\to x'$ a morphism $\m t(f):\m A(f)(y) \to y'$ in a functorial way.
\item The identity type $\Gamma,(x:A),(y:A)\vdash (x=y)$ is represented by the hom-functor $\hom_{\m A}: \m {\Gamma,A} \times_{\m\Gamma} \m{\Gamma,A} \to \mathrm{Gpd}$.
  The corresponding split fibration is the iso-comma category
  \begin{equation*}
    \vcenter{\xymatrix{
        \m{\Gamma,A,A,\mathsf{Id}_A}\ar[r]\ar[d] \ar@{}[dr]|{\cong} &
        \m{\Gamma,A}\ar[d]\\
        \m{\Gamma,A}\ar[r] &
        \m\Gamma
      }}
  \end{equation*}
\end{itemize}
We leave it to the reader to verify that:
\begin{itemize}
\item For $f,g:A\to B$, to give a term $(x:A),(y:A)\vdash (f(x)=g(y))$ means to give a natural isomorphism between the corresponding functors.
\item A groupoid is contractible, in the type-theoretic sense, if and only if it is equivalent to the terminal groupoid, i.e.\ it is nonempty and every two objects are uniquely isomorphic.
\item A groupoid is a set, in the type-theoretic sense, if and only if it is equivalent to a discrete groupoid, i.e.\ any two objects are isomorphic in at most one way.
  We call such groupoids \emph{essential sets}.
\end{itemize}
It is shown in~\cite{hs:gpd-typethy} that the groupoid $\mathsf{Gpd}_\kappa$ of groupoids smaller in cardinality than some inaccessible cardinal $\kappa$ (or more precisely, some small groupoid equivalent to this) supplies a type-theoretic universe in this model.
This universe is \emph{not} univalent, but its sub-universe $\mathsf{ESet}_\kappa$ of essential sets smaller than $\kappa$ \emph{is} univalent.

Thus, the theory of this paper can be interpreted in the groupoid model.
To avoid confusion, in this section we will refer to the type-theoretic notions as \emph{precategories} and \emph{saturated precategories}, reserving the word ``category'' for the ordinary notion of category.

A precategory in the groupoid model consists of a groupoid $A_0$ and a functor $\hom_A:A_0 \times A_0 \to \mathrm{ESet}$, together with suitable composition and identity operations.
Now the Grothendieck construction of $\hom_A$ yields a split fibration $A_1 \to A_0\times A_0$ with essential-set fibers, and the composition and identity operations become functors $m:A_1 \times_{A_0} A_1 \to A_1$ and $i:A_0 \to A_1$.
The associativity and unitality axioms translate into natural isomorphisms
\begin{equation*}
  \vcenter{\xymatrix{
      A_1 \times_{A_0}  A_1 \times_{A_0} A_1 \ar[r]^-{m\times 1}\ar[d]_{1\times m}
      \ar@{}[dr]|\cong &
      A_1 \times_{A_0}  A_1\ar[d]_m\\
      A_1 \times_{A_0}  A_1\ar[r]_-m &
      A_1
    }}
  \qquad\text{and}\qquad
  \vcenter{\xymatrix{
      A_0 \ar[r]^-{i} \ar@{=}[dr] \ar@{}[drr]|(.3){\cong} &
      A_1 \ar[d]^m &
      A_0 \ar[l]_-{i} \ar@{=}[dl] \ar@{}[dll]|(.3){\cong}\\
      & A_0. &
    }}
\end{equation*}
These isomorphisms automatically satisfy the usual coherence axioms, because the fibers of $A_1 \to A_0\times A_0$ are essential sets.
Thus, what we have is precisely a \emph{pseudo double category}~\cite{gp:double-limits}---a weak internal category in categories---with the following additional properties:
\begin{enumerate}
\item The category $A_0$ is a groupoid.\label{item:dc1}
\item The (source, target) functor $A_1 \to A_0\times A_0$ is a fibration.\label{item:dc2}
\item The fibers of $A_1 \to A_0\times A_0$ are essential sets.\label{item:dc3}
\end{enumerate}

Property~\ref{item:dc2} means exactly that this pseudo double category is a \emph{framed bicategory} in the sense of~\cite{shulman:frbi}.
By Appendix C of \textit{ibid.}, to give such a thing is equivalent to giving the category $A_0$, a bicategory $\mathcal{H}A$ with the same objects as $A_0$, and a pseudofunctor $A_0 \to \mathcal{H}A$, which is the identity on objects and such that every morphism in $A_0$ is sent to a left adjoint in $\mathcal{H}A$.
The morphisms of $\mathcal{H}A$ are, of course, the objects of $A_1$, while its 2-morphisms are the morphisms in the fibers of $A_1 \to A_0\times A_0$ (and thus, by~\ref{item:dc3} above, the hom-categories of $\mathcal{H}A$ are essential sets).

The functor $A_0 \to \mathcal{H}A$ acts on a morphism $f:x\to y$ in $A_0$ by lifting the morphism $(1,f):(x,x)\to (x,y)$ in $A_0\times A_0$ with respect to the fibration $A_1 \to A_0\times A_0$, starting at $i(x)$ in the fiber over $(x,x)$.
(In our case, since every morphism in $A_0$ is invertible, it remains so in $\mathcal{H}A$, and thus is automatically a left adjoint.)
Note that with lifting in fibrations of groupoids identified with transport in the type theory, this is equivalent to the operation \idtoiso.

In conclusion, a precategory in the groupoid model can equivalently be identified with the data:
\begin{enumerate}
\item A groupoid $A_0$,
\item A bicategory $\mathcal{H}A$ with the same objects as $A_0$, whose hom-categories are essential sets, and
\item A pseudofunctor $A_0 \to \mathcal{H}A$ which is the identity on objects.
\end{enumerate}
Of course, up to equivalence, a bicategory whose hom-categories are essential sets is just an ordinary category, which may be obtained by passing to isomorphism classes of objects in its hom-categories; let us call this category $\bar A$.
Thus, a precategory in the groupoid model is essentially just an identity-on-objects functor $A_0 \to \bar A$ whose domain is a groupoid.

Finally, the isomorphisms in such a precategory, as defined in \S\ref{sec:cats}, are easily seen to be precisely the equivalences in the bicategory $\mathcal{H}A$, i.e.\ the isomorphisms in the ordinary category $\bar A$.
Thus, to say that this precategory is saturated is to say that for every pair of objects $x,y$, the function
\[ \hom_{A_0}(x,y) \to \mathrm{iso}_{\bar A}(x,y) \]
is a bijection.
In particular, to give a saturated precategory in the groupoid model is equivalent to giving a category in the ordinary sense, $\bar A$.
The groupoid $A_0$ is then canonically identified with the \emph{core} (i.e.\ maximal subgroupoid) of this category $\bar A$.
This provides some semantic justification for the definition of saturated category: it is necessary in order to recover the usual notion of ``category'' in the groupoid model.

Of course, a \emph{strict category} as in \S\ref{sec:strict-categories}, in the groupoid model, corresponds to a precategory as above in which $A_0$ is an essential set.
Unlike the case of a saturated precategory, these data are not obviously determined by the category $\bar A$.
However\dots


\subsection{Simplicial sets}
\label{sec:simplicial-sets}


\subsection{Stacks}
\label{sec:stacks}



%%% Local Variables: 
%%% mode: latex
%%% TeX-master: "hottcats"
%%% End: 


\bibliographystyle{alpha}
\bibliography{hottcats}

\end{document}

%%% Local Variables:
%%% mode: latex
%%% TeX-master: t
%%% End:
