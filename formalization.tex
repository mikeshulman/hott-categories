
\section{The Formalization}
\label{sec:formalization}
% \newcommand{\coq}{\textsf{Coq}}   % this doesn't work properly

Large chunks of the material presented above have been formalized in the proof assistant \textsf{Coq}.
The version of \textsf{Coq} we used is \textsf{Coq} 8.3pl5, patched according to the instructions given by
V.\ Voevodsky \footnote{\url{https://github.com/vladimirias/Foundations/tree/master/Coq_patches}}.

Our formalization is based on Voevodsky's \emph{Foundations} library \cite{vv_foundations}.

Repository, all necessary files --- Voevodsky's, ours --- are on arxiv

TODO get arxiv URL by submitting a preliminary version of this paper

Our general design principles largely follow the conventions established by Voevodsky in the library \cite{vv_foundations} with a few departures. 
The formalization uses only three type constructors: $\Pi$, $\Sigma$, $\textsf{Id}$ and does not use 
almost any of the syntactic sugar of \textsf{Coq} such as the record types. 
Following Voevodsky, we used implicit arguments and, quite extensively, coercions. 
As for the differences, the use of notations, especially with infix symbols (for example, $\textsf{f ;; g}$ for 
the composition of morphisms of a (pre-)category) plays an important role. We also used the section mechanism of \textsf{Coq} 
when several hypotheses were common to a series of constructions and lemmas, e.g., 
when defining an instance of a structure such as an equivalence of precategories.

Since informal type theory, used in the previous section is supposed to match quite closely its formal equivalent, 
the statements of the formalization are very similar to the corresponding statements of the informal type theory. 
For example, TODO: an example. Of course, one example cannot speak for the whole formalization, but we believe it is fairly representative.

The major differences occur when we split a large definition in parts as, for example, for the definition of a precategory. We first, define:

TODO: nice coq code for precategory-ob-mor-pair.

Then we complete the data of a precategory by:

TODO: nice coq code for precategory-data-pair.

And finally, we define a precategory:

TODO: nice coq code for is-precategory and precategory.


\subsection*{Contents of the Formalization}

In this part of the project we aimed on formalizing the Rezk completion together with its universal property. The formalization consists of 10 files:
\begin{itemize}
 \item \texttt{precategories.v} which roughly covers \autoref{sec:cats}.
 \item \texttt{functors-transformations.v} which roughly covers \autoref{sec:transfors}
 \item \texttt{sub-precategories.v} where we define the image of a functor whose target is a category. 
          This is not a separate part of the paper, but it is used (in less generality) in \autoref{thm:rezk-completion}.
 \item \texttt{equivalences.v} where we cover parts of \autoref{sec:equivalences} needed for \autoref{ct:cat-weq-eq}.
 \item \texttt{category-hset.v} where we define the precategory of sets and show that it is a category.
 \item \texttt{yoneda.v} where we cover the main parts of \autoref{ct:yoneda}.
 \item \texttt{whiskering.v} where we define the whiskering, see \autoref{def:whisker}.
 \item \texttt{precomp-fully-faithful.v} that covers \autoref{lem:precomp-faithful} and \ref{lem:precomp-full}.
 \item \texttt{precomp-ess-surj.v} that covers \autoref{ct:cat-weq-eq}.
 \item \texttt{Rezk-completion.v} that puts the previous files together exhibiting \autoref{thm:rezk-completion}.
\end{itemize}
\subsection*{Technical Issues and Statistics}


\subsubsection*{Issues}
Our library has an unsatisfactory aspects due to the design principles outlined above:

Firstly, the packaging of structures, i.e., data types together with operations and properties of these operations,
in $\Sigma$-types, which basically behave like record types,
suffers from the deficiencies outlined in \cite{DBLP:journals/mscs/SpittersW11}. Furthermore,
proof automation using canonical structures \`a la \cite{DBLP:conf/icfp/GonthierZND11} is not
available since we stick to $\Sigma$-types rather than record types.

For instance, a consequence of our restriction to sigma types implies the necessity of defining another
categorical composition in addition to the one of morphisms given by the category itself: a composition of
\emph{iso}morphisms which expresses that isomorphisms are closed under composition.
Using more sophisticated techniques such as type classes, this fact could be established once and for all and the property of
a composition of isomorphisms of being an isomorphism could be established automatically by the system whenever needed.


\subsubsection*{Deviation in the Formalization from the paper definitions}

The formalization deviates very little from the pen--and--paper definitions given in the previous section.
One particular place where deviation is beneficial shall be mentioned here:
in the definition of a left adjoint \ref{TODO:point_to_adjunction}
the mentioned natural transformations $(\epsilon F)(F\eta)$ and $(G\epsilon)(\eta G)$ are actually not
composable; indeed, given a functor $F : A \to B$, we have that $F \circ 1_A \not\equiv F$.
Similarly, $H \circ (G \circ F)\not \equiv (H\circ G)\circ F$. However, \emph{pointwise} these compositions
\emph{are} convertible, we have $(F \circ Id) (a) \equiv F(a)$ and
$\big(H \circ (G \circ F)\big)(a)\equiv \big((H\circ G)\circ F\big)(a)$ for any object $a$, and similar for morphisms, but for morphisms
we do not care here.

TODO : explain why above composition of whiskered natural transformation fails without transport

Thus, in order to make this composition typecheck, we would have to insert coherence equalities

TODO : a better name for coherence equalities

However, if we express the equality of these natural transformations \emph{pointwise} --- which yields an equivalent statement
in the type--theoretic sense of equivalence ---, we can do without the coherence equalities: we have

TODO : explain what we have, give the types


Note that since both statements are propositions and coinhabited, this does not change anything, really, believe me.


\subsubsection*{Statistics}

Our library comprises ten files with ca.\ 1200 lines of specification --- definitions and statements of lemmas and theorems ---
and 2500 lines of proof script altogether.


TODO : coqdoc seems broken in that it produces broken index file, which shows lemmas and definitions multiple times


%   output from coqwc tool, measures lines of code
%
%  spec    proof comments
%       223      297       42 precategories.v
%       287      496       75 functors_transformations.v
%       188      241      107 sub_precategories.v
%       131      252       30 equivalences.v
%        51      125       22 category_hset.v
%        90      147       22 yoneda.v
%        46       42       15 whiskering.v
%        56      292       21 precomp_fully_faithful.v
%       116      814      115 precomp_ess_surj.v
%        39       39       11 rezk_completion.v
%      1227     2745      460 total


%%% Local Variables:
%%% mode: latex
%%% TeX-master: "hottcats"
%%% End:
