
\section{The Formalization}
\label{sec:formalization}
% \newcommand{\coq}{\textsf{Coq}}   % this doesn't work properly

Large chunks of the material presented above have been formalized in the proof assistant \textsf{Coq}.
The version of \textsf{Coq} used is \textsf{Coq} 8.3pl5, patched according to the instructions given by
V.\ Voevodsky\footnote{\url{https://github.com/vladimirias/Foundations/tree/master/Coq_patches}}.

Our formalization is based on Voevodsky's \emph{Foundations} library \cite{vv_foundations},
and is available online\footnote{\url{https://github.com/benediktahrens/rezk_completion}}.


\subsection*{Design principles}
Our general design principles largely follow the conventions established by Voevodsky in the library \cite{vv_foundations} 
with a few departures. 
Both use only three type constructors: $\Pi$, $\Sigma$, $\textsf{Id}$ and do not use 
almost any of the syntactic sugar of \textsf{Coq} such as the record types. 
Following Voevodsky, we use implicit arguments and, quite extensively, coercions. 

We restrict ourselves to these basic type constructors since they have a well-understood
semantics in various homotopy-theoretic models. Implicit arguments and coercions are 
crucial to manage structures of high complexity. Furthermore, they reflect
 familiar mathematical practice.


As for the differences, the use of notations, especially with infix symbols (for example, \lstinline!f ;; g! for 
the composition of morphisms of a precategory) plays an important role in our formalization. 
We also use the section mechanism of \textsf{Coq} 
when several hypotheses are common to a series of constructions and lemmas, e.g., 
when constructing particular examples of complex structures.

\subsection*{Reading the code}

Since informal type theory, used in the previous sections, is supposed to match its formal equivalent quite closely, 
the statements of the formalization are very similar to the corresponding statements of the informal type theory. 
For example, our formal statement correponding to \autoref{def:whisker} looks as follows:
\begin{lstlisting}
Lemma is_nat_trans_pre_whisker (A B C: precategory) (F: functor A B)
   (G H: functor B C) (gamma: nat_trans G H) :
  is_nat_trans (G o F) (H o F) (fun a: ob A => gamma (F a)).
\end{lstlisting}


The major differences occur when we split a large definition in parts as, for example, for the definition of a precategory. 
We first, define:


\begin{lstlisting}
Definition precategory_ob_mor := total2 (
  fun ob : UU => ob -> ob -> hSet).
\end{lstlisting}
%
Given an element \lstinline!C! of the above type, we write \lstinline!ob C!
for its first component and \lstinline!a --> b! for the value of the second 
component on \lstinline!a b : ob C!.

We complete the data of a precategory by:

\begin{lstlisting}
Definition precategory_data := total2 (
   fun C : precategory_ob_mor =>
     dirprod (forall c : ob C, c --> c)
             (forall a b c : ob C, a --> b -> b --> c -> a --> c)).
\end{lstlisting}
In the following we write \lstinline!identity c! for the identity morphism
on an object \lstinline!c! and \lstinline!f ;; g! for the composite of 
morphisms \lstinline!f : a --> b! and \lstinline!g : b --> c!.

We define a predicate expressing that this data constitutes a precategory:
\begin{lstlisting}
Definition is_precategory (C : precategory_data) :=
   dirprod (dirprod (forall (a b : ob C) (f : a --> b),
                         identity a ;; f == f)
                     (forall (a b : ob C) (f : a --> b),
                         f ;; identity b == f))
            (forall (a b c d : ob C)
                    (f : a --> b)(g : b --> c) (h : c --> d),
                     f ;; (g ;; h) == (f ;; g) ;; h).
\end{lstlisting}
As the last step, we say that a precategory is given by the data of a precategory
satisfying the necessary axioms:
\begin{lstlisting}
Definition precategory := total2 is_precategory.
\end{lstlisting}







\subsection*{Contents of the formalization}

In this part of the project we aimed on formalizing the Rezk completion together with its universal property. The formalization consists of 10 files:
\begin{itemize}
 \item \texttt{precategories.v} which roughly covers \autoref{sec:cats}.
 \item \texttt{functors\textunderscore transformations.v} which roughly covers \autoref{sec:transfors}.
 \item \texttt{sub\textunderscore precategories.v} where we define sub-precategories and
                 the image factorization of a functor. 
          This is not a separate part of the paper, but it is used (in less generality) in \autoref{thm:rezk-completion}.
 \item \texttt{equivalences.v} where we cover parts of \autoref{sec:equivalences} needed for \autoref{ct:cat-weq-eq}.
 \item \texttt{category\textunderscore hset.v} where we define the precategory of sets and show that it is a category.
 \item \texttt{yoneda.v} where we cover the main parts of \autoref{ct:yoneda}.
 \item \texttt{whiskering.v} where we define the whiskering, see \autoref{def:whisker}.
 \item \texttt{precomp\textunderscore fully\textunderscore faithful.v} that covers \autoref{lem:precomp-faithful} and \ref{lem:precomp-full}.
 \item \texttt{precomp\textunderscore ess\textunderscore surj.v} that covers \autoref{ct:cat-weq-eq}.
 \item \texttt{Rezk\textunderscore completion.v} that puts the previous files together exhibiting \autoref{thm:rezk-completion}.
\end{itemize}




%\subsubsection*{Issues}
%Our library has an unsatisfactory aspect due to the design principles outlined above.
%Packaging of structures---i.e., data types together with operations and properties of these operations---in 
% $\Sigma$-types often requires proving similar statements several times. For example, we have to define the composition of
% isomorphisms on top of---and in terms of---already existing composition of morphisms in a precategory and then prove 
% some of its properties that have already been proven---or even given as axioms---for their underlying morphisms.
% Using techniques such as type classes, facts of this kind could be established once and for all and various properties 
% of isomorphisms (such as stability under composition) could be inferred automatically by the system whenever needed.


\subsection*{Formalization vs informal definitions}

The formalization deviates very little from the informal definitions given in the previous sections.
We shall mention here the only example of such a deviation, resulting in a slicker definition. 
In \autoref{def:adjoint} the natural transformations $(\epsilon F)$ and $(F\eta)$ (similarly, 
$(G\epsilon)$ and $(\eta G)$) are actually not
composable! We have $\epsilon F : (FG)F \to 1_{B}F$ and $F\eta : F1_A \to F(GF)$. 
However, $(FG)F$ and $F(GF)$ are not convertible, i.e.\ not \emph{definitionally} equal, 
which would be necessary for the composition to typecheck. So in order to state the equality in question
we would have to insert a transport along propositional equality---see \autoref{ct:functor-assoc} and the subsequent discussion.

We overcome this issue by rephrasing the axiom: instead of requiring an equality of natural 
transformations, we require it to hold pointwise. 
These statements are logically and type-theoretically equivalent, but for the latter we have the
desired convertibility: for any $a : A$, the term $\big(F(GF)\big)(a)$ is convertible to $\big((FG)F\big)(a)$.



\subsection*{Statistics}

Our library comprises ten files with ca.\ 180 definitions and 170 lemmas altogether.
The \texttt{coqwc} tool counts 1200 lines of specification --- definitions and statements of lemmas and theorems ---
and 2700 lines of proof script overall.




%%% Local Variables:
%%% mode: latex
%%% TeX-master: "hottcats"
%%% End:
